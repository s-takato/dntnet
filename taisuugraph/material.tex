%%% タイトル material
\documentclass[landscape,10pt]{ujarticle}
\special{papersize=\the\paperwidth,\the\paperheight}
\usepackage{ketpic,ketlayer}
\usepackage{ketslide}
\usepackage{amsmath,amssymb}
\usepackage{bm,enumerate}
\usepackage[dvipdfmx]{graphicx}
\usepackage{color}
\definecolor{slidecolora}{cmyk}{0.98,0.13,0,0.43}
\definecolor{slidecolorb}{cmyk}{0.2,0,0,0}
\definecolor{slidecolorc}{cmyk}{0.2,0,0,0}
\definecolor{slidecolord}{cmyk}{0.2,0,0,0}
\definecolor{slidecolore}{cmyk}{0,0,0,0.5}
\definecolor{slidecolorf}{cmyk}{0,0,0,0.5}
\definecolor{slidecolori}{cmyk}{0.98,0.13,0,0.43}
\def\setthin#1{\def\thin{#1}}
\setthin{0}
\newcommand{\slidepage}[1][s]{%
\setcounter{ketpicctra}{18}%
\if#1m \setcounter{ketpicctra}{1}\fi
\hypersetup{linkcolor=black}%

\begin{layer}{118}{0}
\putnotee{122}{-\theketpicctra.05}{\small\thepage/\pageref{pageend}}
\end{layer}\hypersetup{linkcolor=blue}

}
\usepackage{emath}
\usepackage[dvipdfmx,colorlinks=true,linkcolor=blue,filecolor=blue]{hyperref}
\newcommand{\hako}[4][6]{\fbox{\raisebox{#2 mm}{$\mathstrut$}\raisebox{-#3 mm}{$\mathstrut$}\Ctab{#1 mm}{#4}}}
\def\rad{\;\mathrm{rad}}
\newcommand{\sfrac}[3][0.65]{\scalebox{#1}{$\frac{#2}{#3}$}}

\setmargin{25}{145}{15}{100}

\ketslideinit

\pagestyle{empty}

\begin{document}

\begin{layer}{120}{0}
\putnotese{0}{0}{{\Large\bf
\color[cmyk]{1,1,0,0}

\begin{layer}{120}{0}
{\Huge \putnotes{60}{20}{対数関数のグラフ}}
\end{layer}

}
}
\end{layer}

\def\mainslidetitley{22}
\def\ketcletter{slidecolora}
\def\ketcbox{slidecolorb}
\def\ketdbox{slidecolorc}
\def\ketcframe{slidecolord}
\def\ketcshadow{slidecolore}
\def\ketdshadow{slidecolorf}
\def\slidetitlex{6}
\def\slidetitlesize{1.3}
\def\mketcletter{slidecolori}
\def\mketcbox{yellow}
\def\mketdbox{yellow}
\def\mketcframe{yellow}
\def\mslidetitlex{62}
\def\mslidetitlesize{2}

\color{black}
\Large\bf\boldmath
\addtocounter{page}{-1}

\renewcommand{\hako}[4][-1]{%
\setcounter{ketpicctra}{#2}%
\divide\value{ketpicctra} by 2%
\setcounter{ketpicctrb}{#3}%
\divide\value{ketpicctrb} by 2%
\setcounter{ketpicctrc}{\theketpicctrb}%
\addtocounter{ketpicctrc}{#1}%
\def\kettmp{
\begin{picture}%
(#2, #3)(0,0)%
\settowidth{\Width}{#4}\setlength{\Width}{-0.5\Width}%
\settoheight{\Height}{#4}\settodepth{\Depth}{#4}\setlength{\Height}{-0.5\Height}\setlength{\Depth}{0.5\Depth}\addtolength{\Height}{\Depth}%
\put(\theketpicctra,\theketpicctrb){\hspace*{\Width}\raisebox{\Height}{#4}}%
\end{picture}%
}%
{\unitlength=1mm%
\raisebox{-\theketpicctrc mm}{\fbox{\kettmp}}%
}
}
%%%%%%%%%%%%

%%%%%%%%%%%%%%%%%%%%

\newslide{対数関数$y=\log_a x$のグラフ}

\vspace*{18mm}

\slidepage
\begin{itemize}
\item
対数の意味は \fbox{$y=\log_a x\Longleftrightarrow x=a^y$}
\item
$y=a^x$の$x,y$を交換
\item
$x$軸と$y$軸も入れ替わる
\item
$x$軸が横,$y$軸が縦になるようにする
\item
次のURLをクリックして実行してみよう\\
\normalsize\url{https://s-takato.github.io/dntnet/taisuugraph/sisuu2taisuu.html}
\end{itemize}
%%%%%%%%%%%%

%%%%%%%%%%%%%%%%%%%%


\newslide{指数関数と対数関数}

\vspace*{18mm}

\slidepage

\begin{layer}{120}{0}
\putnotese{63}{17}{\scalebox{0.5}{%%% /polytech.git/n106/fig/taisuu3.tex 
%%% Generator=taisuujs.cdy 
{\unitlength=1cm%
\begin{picture}%
(12,12)(-6,-6)%
\special{pn 8}%
%
\special{pn 4}%
\special{pa -1969  1969}\special{pa -1969 -1969}%
\special{fp}%
\special{pn 8}%
\special{pn 4}%
\special{pa -1575  1969}\special{pa -1575 -1969}%
\special{fp}%
\special{pn 8}%
\special{pn 4}%
\special{pa -1181  1969}\special{pa -1181 -1969}%
\special{fp}%
\special{pn 8}%
\special{pn 4}%
\special{pa  -787  1969}\special{pa  -787 -1969}%
\special{fp}%
\special{pn 8}%
\special{pn 4}%
\special{pa  -394  1969}\special{pa  -394 -1969}%
\special{fp}%
\special{pn 8}%
\special{pn 4}%
\special{pa     0  1969}\special{pa     0 -1969}%
\special{fp}%
\special{pn 8}%
\special{pn 4}%
\special{pa   394  1969}\special{pa   394 -1969}%
\special{fp}%
\special{pn 8}%
\special{pn 4}%
\special{pa   787  1969}\special{pa   787 -1969}%
\special{fp}%
\special{pn 8}%
\special{pn 4}%
\special{pa  1181  1969}\special{pa  1181 -1969}%
\special{fp}%
\special{pn 8}%
\special{pn 4}%
\special{pa  1575  1969}\special{pa  1575 -1969}%
\special{fp}%
\special{pn 8}%
\special{pn 4}%
\special{pa  1969  1969}\special{pa  1969 -1969}%
\special{fp}%
\special{pn 8}%
\special{pn 4}%
\special{pa -1969  1969}\special{pa  1969  1969}%
\special{fp}%
\special{pn 8}%
\special{pn 4}%
\special{pa -1969  1575}\special{pa  1969  1575}%
\special{fp}%
\special{pn 8}%
\special{pn 4}%
\special{pa -1969  1181}\special{pa  1969  1181}%
\special{fp}%
\special{pn 8}%
\special{pn 4}%
\special{pa -1969   787}\special{pa  1969   787}%
\special{fp}%
\special{pn 8}%
\special{pn 4}%
\special{pa -1969   394}\special{pa  1969   394}%
\special{fp}%
\special{pn 8}%
\special{pn 4}%
\special{pa -1969    -0}\special{pa  1969    -0}%
\special{fp}%
\special{pn 8}%
\special{pn 4}%
\special{pa -1969  -394}\special{pa  1969  -394}%
\special{fp}%
\special{pn 8}%
\special{pn 4}%
\special{pa -1969  -787}\special{pa  1969  -787}%
\special{fp}%
\special{pn 8}%
\special{pn 4}%
\special{pa -1969 -1181}\special{pa  1969 -1181}%
\special{fp}%
\special{pn 8}%
\special{pn 4}%
\special{pa -1969 -1575}\special{pa  1969 -1575}%
\special{fp}%
\special{pn 8}%
\special{pn 4}%
\special{pa -1969 -1969}\special{pa  1969 -1969}%
\special{fp}%
\special{pn 8}%
\special{pa -2362 2362}\special{pa -2335 2335}\special{fp}\special{pa -2307 2307}\special{pa -2279 2279}\special{fp}%
\special{pa -2252 2252}\special{pa -2224 2224}\special{fp}\special{pa -2196 2196}\special{pa -2169 2169}\special{fp}%
\special{pa -2141 2141}\special{pa -2114 2114}\special{fp}\special{pa -2086 2086}\special{pa -2058 2058}\special{fp}%
\special{pa -2031 2031}\special{pa -2003 2003}\special{fp}\special{pa -1975 1975}\special{pa -1948 1948}\special{fp}%
\special{pa -1920 1920}\special{pa -1893 1893}\special{fp}\special{pa -1865 1865}\special{pa -1837 1837}\special{fp}%
\special{pa -1810 1810}\special{pa -1782 1782}\special{fp}\special{pa -1754 1754}\special{pa -1727 1727}\special{fp}%
\special{pa -1699 1699}\special{pa -1672 1672}\special{fp}\special{pa -1644 1644}\special{pa -1616 1616}\special{fp}%
\special{pa -1589 1589}\special{pa -1561 1561}\special{fp}\special{pa -1533 1533}\special{pa -1506 1506}\special{fp}%
\special{pa -1478 1478}\special{pa -1450 1450}\special{fp}\special{pa -1423 1423}\special{pa -1395 1395}\special{fp}%
\special{pa -1368 1368}\special{pa -1340 1340}\special{fp}\special{pa -1312 1312}\special{pa -1285 1285}\special{fp}%
\special{pa -1257 1257}\special{pa -1229 1229}\special{fp}\special{pa -1202 1202}\special{pa -1174 1174}\special{fp}%
\special{pa -1147 1147}\special{pa -1119 1119}\special{fp}\special{pa -1091 1091}\special{pa -1064 1064}\special{fp}%
\special{pa -1036 1036}\special{pa -1008 1008}\special{fp}\special{pa -981 981}\special{pa -953 953}\special{fp}%
\special{pa -926 926}\special{pa -898 898}\special{fp}\special{pa -870 870}\special{pa -843 843}\special{fp}%
\special{pa -815 815}\special{pa -787 787}\special{fp}\special{pa -760 760}\special{pa -732 732}\special{fp}%
\special{pa -705 705}\special{pa -677 677}\special{fp}\special{pa -649 649}\special{pa -622 622}\special{fp}%
\special{pa -594 594}\special{pa -566 566}\special{fp}\special{pa -539 539}\special{pa -511 511}\special{fp}%
\special{pa -483 483}\special{pa -456 456}\special{fp}\special{pa -428 428}\special{pa -401 401}\special{fp}%
\special{pa -373 373}\special{pa -345 345}\special{fp}\special{pa -318 318}\special{pa -290 290}\special{fp}%
\special{pa -262 262}\special{pa -235 235}\special{fp}\special{pa -207 207}\special{pa -180 180}\special{fp}%
\special{pa -152 152}\special{pa -124 124}\special{fp}\special{pa -97 97}\special{pa -69 69}\special{fp}%
\special{pa -41 41}\special{pa -14 14}\special{fp}\special{pa 14 -14}\special{pa 41 -41}\special{fp}%
\special{pa 69 -69}\special{pa 97 -97}\special{fp}\special{pa 124 -124}\special{pa 152 -152}\special{fp}%
\special{pa 180 -180}\special{pa 207 -207}\special{fp}\special{pa 235 -235}\special{pa 262 -262}\special{fp}%
\special{pa 290 -290}\special{pa 318 -318}\special{fp}\special{pa 345 -345}\special{pa 373 -373}\special{fp}%
\special{pa 401 -401}\special{pa 428 -428}\special{fp}\special{pa 456 -456}\special{pa 483 -483}\special{fp}%
\special{pa 511 -511}\special{pa 539 -539}\special{fp}\special{pa 566 -566}\special{pa 594 -594}\special{fp}%
\special{pa 622 -622}\special{pa 649 -649}\special{fp}\special{pa 677 -677}\special{pa 705 -705}\special{fp}%
\special{pa 732 -732}\special{pa 760 -760}\special{fp}\special{pa 787 -787}\special{pa 815 -815}\special{fp}%
\special{pa 843 -843}\special{pa 870 -870}\special{fp}\special{pa 898 -898}\special{pa 926 -926}\special{fp}%
\special{pa 953 -953}\special{pa 981 -981}\special{fp}\special{pa 1008 -1008}\special{pa 1036 -1036}\special{fp}%
\special{pa 1064 -1064}\special{pa 1091 -1091}\special{fp}\special{pa 1119 -1119}\special{pa 1147 -1147}\special{fp}%
\special{pa 1174 -1174}\special{pa 1202 -1202}\special{fp}\special{pa 1229 -1229}\special{pa 1257 -1257}\special{fp}%
\special{pa 1285 -1285}\special{pa 1312 -1312}\special{fp}\special{pa 1340 -1340}\special{pa 1368 -1368}\special{fp}%
\special{pa 1395 -1395}\special{pa 1423 -1423}\special{fp}\special{pa 1450 -1450}\special{pa 1478 -1478}\special{fp}%
\special{pa 1506 -1506}\special{pa 1533 -1533}\special{fp}\special{pa 1561 -1561}\special{pa 1589 -1589}\special{fp}%
\special{pa 1616 -1616}\special{pa 1644 -1644}\special{fp}\special{pa 1672 -1672}\special{pa 1699 -1699}\special{fp}%
\special{pa 1727 -1727}\special{pa 1754 -1754}\special{fp}\special{pa 1782 -1782}\special{pa 1810 -1810}\special{fp}%
\special{pa 1837 -1837}\special{pa 1865 -1865}\special{fp}\special{pa 1893 -1893}\special{pa 1920 -1920}\special{fp}%
\special{pa 1948 -1948}\special{pa 1975 -1975}\special{fp}\special{pa 2003 -2003}\special{pa 2031 -2031}\special{fp}%
\special{pa 2058 -2058}\special{pa 2086 -2086}\special{fp}\special{pa 2114 -2114}\special{pa 2141 -2141}\special{fp}%
\special{pa 2169 -2169}\special{pa 2196 -2196}\special{fp}\special{pa 2224 -2224}\special{pa 2252 -2252}\special{fp}%
\special{pa 2279 -2279}\special{pa 2307 -2307}\special{fp}\special{pa 2335 -2335}\special{pa 2362 -2362}\special{fp}%
%
%
{%
\color[cmyk]{1,0,0,0}%
\special{pn 12}%
\special{pa -2362    -1}\special{pa -2315    -1}\special{pa -2268    -1}\special{pa -2220    -1}%
\special{pa -2173    -1}\special{pa -2126    -1}\special{pa -2079    -1}\special{pa -2031    -1}%
\special{pa -1984    -2}\special{pa -1937    -2}\special{pa -1890    -2}\special{pa -1843    -2}%
\special{pa -1795    -3}\special{pa -1748    -3}\special{pa -1701    -3}\special{pa -1654    -4}%
\special{pa -1606    -4}\special{pa -1559    -5}\special{pa -1512    -6}\special{pa -1465    -7}%
\special{pa -1417    -8}\special{pa -1370    -9}\special{pa -1323   -10}\special{pa -1276   -11}%
\special{pa -1228   -13}\special{pa -1181   -15}\special{pa -1134   -17}\special{pa -1087   -19}%
\special{pa -1039   -22}\special{pa  -992   -25}\special{pa  -945   -28}\special{pa  -898   -32}%
\special{pa  -850   -37}\special{pa  -803   -42}\special{pa  -756   -48}\special{pa  -709   -54}%
\special{pa  -661   -62}\special{pa  -614   -71}\special{pa  -567   -81}\special{pa  -520   -92}%
\special{pa  -472  -105}\special{pa  -425  -120}\special{pa  -378  -137}\special{pa  -331  -156}%
\special{pa  -283  -179}\special{pa  -236  -204}\special{pa  -189  -232}\special{pa  -142  -265}%
\special{pa   -94  -302}\special{pa   -47  -345}\special{pa     0  -394}\special{pa    47  -449}%
\special{pa    94  -512}\special{pa   142  -585}\special{pa   189  -667}\special{pa   236  -761}%
\special{pa   283  -868}\special{pa   331  -991}\special{pa   378 -1130}\special{pa   425 -1290}%
\special{pa   472 -1471}\special{pa   520 -1679}\special{pa   567 -1915}\special{pa   614 -2185}%
\special{pa   641 -2362}%
\special{fp}%
\special{pn 8}%
}%
\special{pn 12}%
\special{pa     1  2362}\special{pa     1  2315}\special{pa     1  2268}\special{pa     1  2220}%
\special{pa     1  2173}\special{pa     1  2126}\special{pa     1  2079}\special{pa     1  2031}%
\special{pa     2  1984}\special{pa     2  1937}\special{pa     2  1890}\special{pa     2  1843}%
\special{pa     3  1795}\special{pa     3  1748}\special{pa     3  1701}\special{pa     4  1654}%
\special{pa     4  1606}\special{pa     5  1559}\special{pa     6  1512}\special{pa     7  1465}%
\special{pa     8  1417}\special{pa     9  1370}\special{pa    10  1323}\special{pa    11  1276}%
\special{pa    13  1228}\special{pa    15  1181}\special{pa    17  1134}\special{pa    19  1087}%
\special{pa    22  1039}\special{pa    25   992}\special{pa    28   945}\special{pa    32   898}%
\special{pa    37   850}\special{pa    42   803}\special{pa    48   756}\special{pa    54   709}%
\special{pa    62   661}\special{pa    71   614}\special{pa    81   567}\special{pa    92   520}%
\special{pa   105   472}\special{pa   120   425}\special{pa   137   378}\special{pa   156   331}%
\special{pa   179   283}\special{pa   204   236}\special{pa   232   189}\special{pa   265   142}%
\special{pa   302    94}\special{pa   345    47}\special{pa   394    -0}\special{pa   449   -47}%
\special{pa   512   -94}\special{pa   585  -142}\special{pa   667  -189}\special{pa   761  -236}%
\special{pa   868  -283}\special{pa   991  -331}\special{pa  1130  -378}\special{pa  1290  -425}%
\special{pa  1471  -472}\special{pa  1679  -520}\special{pa  1915  -567}\special{pa  2185  -614}%
\special{pa  2362  -641}%
\special{fp}%
\special{pn 8}%
\settowidth{\Width}{$y=\log_{3}x$}\setlength{\Width}{0\Width}%
\settoheight{\Height}{$y=\log_{3}x$}\settodepth{\Depth}{$y=\log_{3}x$}\setlength{\Height}{-0.5\Height}\setlength{\Depth}{0.5\Depth}\addtolength{\Height}{\Depth}%
\put(5.6000000,-0.5000000){\hspace*{\Width}\raisebox{\Height}{$y=\log_{3}x$}}%
%
\settowidth{\Width}{$y=3^x$}\setlength{\Width}{-1\Width}%
\settoheight{\Height}{$y=3^x$}\settodepth{\Depth}{$y=3^x$}\setlength{\Height}{-0.5\Height}\setlength{\Depth}{0.5\Depth}\addtolength{\Height}{\Depth}%
\put(-0.6000000,5.5000000){\hspace*{\Width}\raisebox{\Height}{$y=3^x$}}%
%
\special{pa -1969   -20}\special{pa -1969    20}%
\special{fp}%
\settowidth{\Width}{$-5$}\setlength{\Width}{-0.5\Width}%
\settoheight{\Height}{$-5$}\settodepth{\Depth}{$-5$}\setlength{\Height}{-\Height}%
\put(-5.0000000,-0.1000000){\hspace*{\Width}\raisebox{\Height}{$-5$}}%
%
\special{pa -1575   -20}\special{pa -1575    20}%
\special{fp}%
\settowidth{\Width}{$-4$}\setlength{\Width}{-0.5\Width}%
\settoheight{\Height}{$-4$}\settodepth{\Depth}{$-4$}\setlength{\Height}{-\Height}%
\put(-4.0000000,-0.1000000){\hspace*{\Width}\raisebox{\Height}{$-4$}}%
%
\special{pa -1181   -20}\special{pa -1181    20}%
\special{fp}%
\settowidth{\Width}{$-3$}\setlength{\Width}{-0.5\Width}%
\settoheight{\Height}{$-3$}\settodepth{\Depth}{$-3$}\setlength{\Height}{-\Height}%
\put(-3.0000000,-0.1000000){\hspace*{\Width}\raisebox{\Height}{$-3$}}%
%
\special{pa  -787   -20}\special{pa  -787    20}%
\special{fp}%
\settowidth{\Width}{$-2$}\setlength{\Width}{-0.5\Width}%
\settoheight{\Height}{$-2$}\settodepth{\Depth}{$-2$}\setlength{\Height}{-\Height}%
\put(-2.0000000,-0.1000000){\hspace*{\Width}\raisebox{\Height}{$-2$}}%
%
\special{pa  -394   -20}\special{pa  -394    20}%
\special{fp}%
\settowidth{\Width}{$-1$}\setlength{\Width}{-0.5\Width}%
\settoheight{\Height}{$-1$}\settodepth{\Depth}{$-1$}\setlength{\Height}{-\Height}%
\put(-1.0000000,-0.1000000){\hspace*{\Width}\raisebox{\Height}{$-1$}}%
%
\special{pa   394   -20}\special{pa   394    20}%
\special{fp}%
\settowidth{\Width}{$1$}\setlength{\Width}{-0.5\Width}%
\settoheight{\Height}{$1$}\settodepth{\Depth}{$1$}\setlength{\Height}{-\Height}%
\put(1.0000000,-0.1000000){\hspace*{\Width}\raisebox{\Height}{$1$}}%
%
\special{pa   787   -20}\special{pa   787    20}%
\special{fp}%
\settowidth{\Width}{$2$}\setlength{\Width}{-0.5\Width}%
\settoheight{\Height}{$2$}\settodepth{\Depth}{$2$}\setlength{\Height}{-\Height}%
\put(2.0000000,-0.1000000){\hspace*{\Width}\raisebox{\Height}{$2$}}%
%
\special{pa  1181   -20}\special{pa  1181    20}%
\special{fp}%
\settowidth{\Width}{$3$}\setlength{\Width}{-0.5\Width}%
\settoheight{\Height}{$3$}\settodepth{\Depth}{$3$}\setlength{\Height}{-\Height}%
\put(3.0000000,-0.1000000){\hspace*{\Width}\raisebox{\Height}{$3$}}%
%
\special{pa  1575   -20}\special{pa  1575    20}%
\special{fp}%
\settowidth{\Width}{$4$}\setlength{\Width}{-0.5\Width}%
\settoheight{\Height}{$4$}\settodepth{\Depth}{$4$}\setlength{\Height}{-\Height}%
\put(4.0000000,-0.1000000){\hspace*{\Width}\raisebox{\Height}{$4$}}%
%
\special{pa  1969   -20}\special{pa  1969    20}%
\special{fp}%
\settowidth{\Width}{$5$}\setlength{\Width}{-0.5\Width}%
\settoheight{\Height}{$5$}\settodepth{\Depth}{$5$}\setlength{\Height}{-\Height}%
\put(5.0000000,-0.1000000){\hspace*{\Width}\raisebox{\Height}{$5$}}%
%
\special{pa    20  1969}\special{pa   -20  1969}%
\special{fp}%
\settowidth{\Width}{$-5$}\setlength{\Width}{-1\Width}%
\settoheight{\Height}{$-5$}\settodepth{\Depth}{$-5$}\setlength{\Height}{-0.5\Height}\setlength{\Depth}{0.5\Depth}\addtolength{\Height}{\Depth}%
\put(-0.1000000,-5.0000000){\hspace*{\Width}\raisebox{\Height}{$-5$}}%
%
\special{pa    20  1575}\special{pa   -20  1575}%
\special{fp}%
\settowidth{\Width}{$-4$}\setlength{\Width}{-1\Width}%
\settoheight{\Height}{$-4$}\settodepth{\Depth}{$-4$}\setlength{\Height}{-0.5\Height}\setlength{\Depth}{0.5\Depth}\addtolength{\Height}{\Depth}%
\put(-0.1000000,-4.0000000){\hspace*{\Width}\raisebox{\Height}{$-4$}}%
%
\special{pa    20  1181}\special{pa   -20  1181}%
\special{fp}%
\settowidth{\Width}{$-3$}\setlength{\Width}{-1\Width}%
\settoheight{\Height}{$-3$}\settodepth{\Depth}{$-3$}\setlength{\Height}{-0.5\Height}\setlength{\Depth}{0.5\Depth}\addtolength{\Height}{\Depth}%
\put(-0.1000000,-3.0000000){\hspace*{\Width}\raisebox{\Height}{$-3$}}%
%
\special{pa    20   787}\special{pa   -20   787}%
\special{fp}%
\settowidth{\Width}{$-2$}\setlength{\Width}{-1\Width}%
\settoheight{\Height}{$-2$}\settodepth{\Depth}{$-2$}\setlength{\Height}{-0.5\Height}\setlength{\Depth}{0.5\Depth}\addtolength{\Height}{\Depth}%
\put(-0.1000000,-2.0000000){\hspace*{\Width}\raisebox{\Height}{$-2$}}%
%
\special{pa    20   394}\special{pa   -20   394}%
\special{fp}%
\settowidth{\Width}{$-1$}\setlength{\Width}{-1\Width}%
\settoheight{\Height}{$-1$}\settodepth{\Depth}{$-1$}\setlength{\Height}{-0.5\Height}\setlength{\Depth}{0.5\Depth}\addtolength{\Height}{\Depth}%
\put(-0.1000000,-1.0000000){\hspace*{\Width}\raisebox{\Height}{$-1$}}%
%
\special{pa    20    -0}\special{pa   -20    -0}%
\special{fp}%
\settowidth{\Width}{$0$}\setlength{\Width}{-1\Width}%
\settoheight{\Height}{$0$}\settodepth{\Depth}{$0$}\setlength{\Height}{-0.5\Height}\setlength{\Depth}{0.5\Depth}\addtolength{\Height}{\Depth}%
\put(-0.1000000,0.0000000){\hspace*{\Width}\raisebox{\Height}{$0$}}%
%
\special{pa    20  -394}\special{pa   -20  -394}%
\special{fp}%
\settowidth{\Width}{$1$}\setlength{\Width}{-1\Width}%
\settoheight{\Height}{$1$}\settodepth{\Depth}{$1$}\setlength{\Height}{-0.5\Height}\setlength{\Depth}{0.5\Depth}\addtolength{\Height}{\Depth}%
\put(-0.1000000,1.0000000){\hspace*{\Width}\raisebox{\Height}{$1$}}%
%
\special{pa    20  -787}\special{pa   -20  -787}%
\special{fp}%
\settowidth{\Width}{$2$}\setlength{\Width}{-1\Width}%
\settoheight{\Height}{$2$}\settodepth{\Depth}{$2$}\setlength{\Height}{-0.5\Height}\setlength{\Depth}{0.5\Depth}\addtolength{\Height}{\Depth}%
\put(-0.1000000,2.0000000){\hspace*{\Width}\raisebox{\Height}{$2$}}%
%
\special{pa    20 -1181}\special{pa   -20 -1181}%
\special{fp}%
\settowidth{\Width}{$3$}\setlength{\Width}{-1\Width}%
\settoheight{\Height}{$3$}\settodepth{\Depth}{$3$}\setlength{\Height}{-0.5\Height}\setlength{\Depth}{0.5\Depth}\addtolength{\Height}{\Depth}%
\put(-0.1000000,3.0000000){\hspace*{\Width}\raisebox{\Height}{$3$}}%
%
\special{pa    20 -1575}\special{pa   -20 -1575}%
\special{fp}%
\settowidth{\Width}{$4$}\setlength{\Width}{-1\Width}%
\settoheight{\Height}{$4$}\settodepth{\Depth}{$4$}\setlength{\Height}{-0.5\Height}\setlength{\Depth}{0.5\Depth}\addtolength{\Height}{\Depth}%
\put(-0.1000000,4.0000000){\hspace*{\Width}\raisebox{\Height}{$4$}}%
%
\special{pa    20 -1969}\special{pa   -20 -1969}%
\special{fp}%
\settowidth{\Width}{$5$}\setlength{\Width}{-1\Width}%
\settoheight{\Height}{$5$}\settodepth{\Depth}{$5$}\setlength{\Height}{-0.5\Height}\setlength{\Depth}{0.5\Depth}\addtolength{\Height}{\Depth}%
\put(-0.1000000,5.0000000){\hspace*{\Width}\raisebox{\Height}{$5$}}%
%
\special{pa -2362    -0}\special{pa  2362    -0}%
\special{fp}%
\special{pa     0  2362}\special{pa     0 -2362}%
\special{fp}%
\settowidth{\Width}{$x$}\setlength{\Width}{0\Width}%
\settoheight{\Height}{$x$}\settodepth{\Depth}{$x$}\setlength{\Height}{-0.5\Height}\setlength{\Depth}{0.5\Depth}\addtolength{\Height}{\Depth}%
\put(6.0500000,0.0000000){\hspace*{\Width}\raisebox{\Height}{$x$}}%
%
\settowidth{\Width}{$y$}\setlength{\Width}{-0.5\Width}%
\settoheight{\Height}{$y$}\settodepth{\Depth}{$y$}\setlength{\Height}{\Depth}%
\put(0.0000000,6.0500000){\hspace*{\Width}\raisebox{\Height}{$y$}}%
%
\settowidth{\Width}{O}\setlength{\Width}{-1\Width}%
\settoheight{\Height}{O}\settodepth{\Depth}{O}\setlength{\Height}{-\Height}%
\put(-0.0500000,-0.0500000){\hspace*{\Width}\raisebox{\Height}{O}}%
%
\end{picture}}%}}
\end{layer}

\begin{itemize}
\item
{\normalsize\url{https://s-takato.github.io/dntnet/taisuugraph/taisuu.html}}
\item
$y=a^x$と$y=\log_a x$は\\$y=x$に関して対称
\item
このような関数を\\{\color{red}逆関数}という
\item
対数関数と指数関数は\\
\hspace*{2zw}逆関数どうし
\end{itemize}

\sameslide

\vspace*{18mm}

\slidepage

\begin{layer}{120}{0}
\putnotese{63}{17}{\scalebox{0.5}{%%% /polytech.git/n106/fig/taisuu5.tex 
%%% Generator=taisuujs.cdy 
{\unitlength=1cm%
\begin{picture}%
(12,12)(-6,-6)%
\special{pn 8}%
%
\special{pn 4}%
\special{pa -1969  1969}\special{pa -1969 -1969}%
\special{fp}%
\special{pn 8}%
\special{pn 4}%
\special{pa -1575  1969}\special{pa -1575 -1969}%
\special{fp}%
\special{pn 8}%
\special{pn 4}%
\special{pa -1181  1969}\special{pa -1181 -1969}%
\special{fp}%
\special{pn 8}%
\special{pn 4}%
\special{pa  -787  1969}\special{pa  -787 -1969}%
\special{fp}%
\special{pn 8}%
\special{pn 4}%
\special{pa  -394  1969}\special{pa  -394 -1969}%
\special{fp}%
\special{pn 8}%
\special{pn 4}%
\special{pa     0  1969}\special{pa     0 -1969}%
\special{fp}%
\special{pn 8}%
\special{pn 4}%
\special{pa   394  1969}\special{pa   394 -1969}%
\special{fp}%
\special{pn 8}%
\special{pn 4}%
\special{pa   787  1969}\special{pa   787 -1969}%
\special{fp}%
\special{pn 8}%
\special{pn 4}%
\special{pa  1181  1969}\special{pa  1181 -1969}%
\special{fp}%
\special{pn 8}%
\special{pn 4}%
\special{pa  1575  1969}\special{pa  1575 -1969}%
\special{fp}%
\special{pn 8}%
\special{pn 4}%
\special{pa  1969  1969}\special{pa  1969 -1969}%
\special{fp}%
\special{pn 8}%
\special{pn 4}%
\special{pa -1969  1969}\special{pa  1969  1969}%
\special{fp}%
\special{pn 8}%
\special{pn 4}%
\special{pa -1969  1575}\special{pa  1969  1575}%
\special{fp}%
\special{pn 8}%
\special{pn 4}%
\special{pa -1969  1181}\special{pa  1969  1181}%
\special{fp}%
\special{pn 8}%
\special{pn 4}%
\special{pa -1969   787}\special{pa  1969   787}%
\special{fp}%
\special{pn 8}%
\special{pn 4}%
\special{pa -1969   394}\special{pa  1969   394}%
\special{fp}%
\special{pn 8}%
\special{pn 4}%
\special{pa -1969    -0}\special{pa  1969    -0}%
\special{fp}%
\special{pn 8}%
\special{pn 4}%
\special{pa -1969  -394}\special{pa  1969  -394}%
\special{fp}%
\special{pn 8}%
\special{pn 4}%
\special{pa -1969  -787}\special{pa  1969  -787}%
\special{fp}%
\special{pn 8}%
\special{pn 4}%
\special{pa -1969 -1181}\special{pa  1969 -1181}%
\special{fp}%
\special{pn 8}%
\special{pn 4}%
\special{pa -1969 -1575}\special{pa  1969 -1575}%
\special{fp}%
\special{pn 8}%
\special{pn 4}%
\special{pa -1969 -1969}\special{pa  1969 -1969}%
\special{fp}%
\special{pn 8}%
\special{pa -2362 2362}\special{pa -2335 2335}\special{fp}\special{pa -2307 2307}\special{pa -2279 2279}\special{fp}%
\special{pa -2252 2252}\special{pa -2224 2224}\special{fp}\special{pa -2196 2196}\special{pa -2169 2169}\special{fp}%
\special{pa -2141 2141}\special{pa -2114 2114}\special{fp}\special{pa -2086 2086}\special{pa -2058 2058}\special{fp}%
\special{pa -2031 2031}\special{pa -2003 2003}\special{fp}\special{pa -1975 1975}\special{pa -1948 1948}\special{fp}%
\special{pa -1920 1920}\special{pa -1893 1893}\special{fp}\special{pa -1865 1865}\special{pa -1837 1837}\special{fp}%
\special{pa -1810 1810}\special{pa -1782 1782}\special{fp}\special{pa -1754 1754}\special{pa -1727 1727}\special{fp}%
\special{pa -1699 1699}\special{pa -1672 1672}\special{fp}\special{pa -1644 1644}\special{pa -1616 1616}\special{fp}%
\special{pa -1589 1589}\special{pa -1561 1561}\special{fp}\special{pa -1533 1533}\special{pa -1506 1506}\special{fp}%
\special{pa -1478 1478}\special{pa -1450 1450}\special{fp}\special{pa -1423 1423}\special{pa -1395 1395}\special{fp}%
\special{pa -1368 1368}\special{pa -1340 1340}\special{fp}\special{pa -1312 1312}\special{pa -1285 1285}\special{fp}%
\special{pa -1257 1257}\special{pa -1229 1229}\special{fp}\special{pa -1202 1202}\special{pa -1174 1174}\special{fp}%
\special{pa -1147 1147}\special{pa -1119 1119}\special{fp}\special{pa -1091 1091}\special{pa -1064 1064}\special{fp}%
\special{pa -1036 1036}\special{pa -1008 1008}\special{fp}\special{pa -981 981}\special{pa -953 953}\special{fp}%
\special{pa -926 926}\special{pa -898 898}\special{fp}\special{pa -870 870}\special{pa -843 843}\special{fp}%
\special{pa -815 815}\special{pa -787 787}\special{fp}\special{pa -760 760}\special{pa -732 732}\special{fp}%
\special{pa -705 705}\special{pa -677 677}\special{fp}\special{pa -649 649}\special{pa -622 622}\special{fp}%
\special{pa -594 594}\special{pa -566 566}\special{fp}\special{pa -539 539}\special{pa -511 511}\special{fp}%
\special{pa -483 483}\special{pa -456 456}\special{fp}\special{pa -428 428}\special{pa -401 401}\special{fp}%
\special{pa -373 373}\special{pa -345 345}\special{fp}\special{pa -318 318}\special{pa -290 290}\special{fp}%
\special{pa -262 262}\special{pa -235 235}\special{fp}\special{pa -207 207}\special{pa -180 180}\special{fp}%
\special{pa -152 152}\special{pa -124 124}\special{fp}\special{pa -97 97}\special{pa -69 69}\special{fp}%
\special{pa -41 41}\special{pa -14 14}\special{fp}\special{pa 14 -14}\special{pa 41 -41}\special{fp}%
\special{pa 69 -69}\special{pa 97 -97}\special{fp}\special{pa 124 -124}\special{pa 152 -152}\special{fp}%
\special{pa 180 -180}\special{pa 207 -207}\special{fp}\special{pa 235 -235}\special{pa 262 -262}\special{fp}%
\special{pa 290 -290}\special{pa 318 -318}\special{fp}\special{pa 345 -345}\special{pa 373 -373}\special{fp}%
\special{pa 401 -401}\special{pa 428 -428}\special{fp}\special{pa 456 -456}\special{pa 483 -483}\special{fp}%
\special{pa 511 -511}\special{pa 539 -539}\special{fp}\special{pa 566 -566}\special{pa 594 -594}\special{fp}%
\special{pa 622 -622}\special{pa 649 -649}\special{fp}\special{pa 677 -677}\special{pa 705 -705}\special{fp}%
\special{pa 732 -732}\special{pa 760 -760}\special{fp}\special{pa 787 -787}\special{pa 815 -815}\special{fp}%
\special{pa 843 -843}\special{pa 870 -870}\special{fp}\special{pa 898 -898}\special{pa 926 -926}\special{fp}%
\special{pa 953 -953}\special{pa 981 -981}\special{fp}\special{pa 1008 -1008}\special{pa 1036 -1036}\special{fp}%
\special{pa 1064 -1064}\special{pa 1091 -1091}\special{fp}\special{pa 1119 -1119}\special{pa 1147 -1147}\special{fp}%
\special{pa 1174 -1174}\special{pa 1202 -1202}\special{fp}\special{pa 1229 -1229}\special{pa 1257 -1257}\special{fp}%
\special{pa 1285 -1285}\special{pa 1312 -1312}\special{fp}\special{pa 1340 -1340}\special{pa 1368 -1368}\special{fp}%
\special{pa 1395 -1395}\special{pa 1423 -1423}\special{fp}\special{pa 1450 -1450}\special{pa 1478 -1478}\special{fp}%
\special{pa 1506 -1506}\special{pa 1533 -1533}\special{fp}\special{pa 1561 -1561}\special{pa 1589 -1589}\special{fp}%
\special{pa 1616 -1616}\special{pa 1644 -1644}\special{fp}\special{pa 1672 -1672}\special{pa 1699 -1699}\special{fp}%
\special{pa 1727 -1727}\special{pa 1754 -1754}\special{fp}\special{pa 1782 -1782}\special{pa 1810 -1810}\special{fp}%
\special{pa 1837 -1837}\special{pa 1865 -1865}\special{fp}\special{pa 1893 -1893}\special{pa 1920 -1920}\special{fp}%
\special{pa 1948 -1948}\special{pa 1975 -1975}\special{fp}\special{pa 2003 -2003}\special{pa 2031 -2031}\special{fp}%
\special{pa 2058 -2058}\special{pa 2086 -2086}\special{fp}\special{pa 2114 -2114}\special{pa 2141 -2141}\special{fp}%
\special{pa 2169 -2169}\special{pa 2196 -2196}\special{fp}\special{pa 2224 -2224}\special{pa 2252 -2252}\special{fp}%
\special{pa 2279 -2279}\special{pa 2307 -2307}\special{fp}\special{pa 2335 -2335}\special{pa 2362 -2362}\special{fp}%
%
%
{%
\color[cmyk]{1,0,0,0}%
\special{pn 12}%
\special{pa -2362    -0}\special{pa -2315    -0}\special{pa -2268    -0}\special{pa -2220    -0}%
\special{pa -2173    -0}\special{pa -2126    -0}\special{pa -2079    -0}\special{pa -2031    -0}%
\special{pa -1984    -0}\special{pa -1937    -0}\special{pa -1890    -0}\special{pa -1843    -0}%
\special{pa -1795    -0}\special{pa -1748    -0}\special{pa -1701    -0}\special{pa -1654    -0}%
\special{pa -1606    -1}\special{pa -1559    -1}\special{pa -1512    -1}\special{pa -1465    -1}%
\special{pa -1417    -1}\special{pa -1370    -1}\special{pa -1323    -2}\special{pa -1276    -2}%
\special{pa -1228    -3}\special{pa -1181    -3}\special{pa -1134    -4}\special{pa -1087    -5}%
\special{pa -1039    -6}\special{pa  -992    -7}\special{pa  -945    -8}\special{pa  -898   -10}%
\special{pa  -850   -12}\special{pa  -803   -15}\special{pa  -756   -18}\special{pa  -709   -22}%
\special{pa  -661   -26}\special{pa  -614   -32}\special{pa  -567   -39}\special{pa  -520   -47}%
\special{pa  -472   -57}\special{pa  -425   -69}\special{pa  -378   -84}\special{pa  -331  -102}%
\special{pa  -283  -124}\special{pa  -236  -150}\special{pa  -189  -182}\special{pa  -142  -221}%
\special{pa   -94  -268}\special{pa   -47  -325}\special{pa     0  -394}\special{pa    47  -478}%
\special{pa    94  -579}\special{pa   142  -703}\special{pa   189  -852}\special{pa   236 -1034}%
\special{pa   283 -1254}\special{pa   331 -1522}\special{pa   378 -1846}\special{pa   425 -2239}%
\special{pa   437 -2362}%
\special{fp}%
\special{pn 8}%
}%
\special{pn 12}%
\special{pa     0  2362}\special{pa     0  2315}\special{pa     0  2268}\special{pa     0  2220}%
\special{pa     0  2173}\special{pa     0  2126}\special{pa     0  2079}\special{pa     0  2031}%
\special{pa     0  1984}\special{pa     0  1937}\special{pa     0  1890}\special{pa     0  1843}%
\special{pa     0  1795}\special{pa     0  1748}\special{pa     0  1701}\special{pa     0  1654}%
\special{pa     1  1606}\special{pa     1  1559}\special{pa     1  1512}\special{pa     1  1465}%
\special{pa     1  1417}\special{pa     1  1370}\special{pa     2  1323}\special{pa     2  1276}%
\special{pa     3  1228}\special{pa     3  1181}\special{pa     4  1134}\special{pa     5  1087}%
\special{pa     6  1039}\special{pa     7   992}\special{pa     8   945}\special{pa    10   898}%
\special{pa    12   850}\special{pa    15   803}\special{pa    18   756}\special{pa    22   709}%
\special{pa    26   661}\special{pa    32   614}\special{pa    39   567}\special{pa    47   520}%
\special{pa    57   472}\special{pa    69   425}\special{pa    84   378}\special{pa   102   331}%
\special{pa   124   283}\special{pa   150   236}\special{pa   182   189}\special{pa   221   142}%
\special{pa   268    94}\special{pa   325    47}\special{pa   394    -0}\special{pa   478   -47}%
\special{pa   579   -94}\special{pa   703  -142}\special{pa   852  -189}\special{pa  1034  -236}%
\special{pa  1254  -283}\special{pa  1522  -331}\special{pa  1846  -378}\special{pa  2239  -425}%
\special{pa  2362  -437}%
\special{fp}%
\special{pn 8}%
\settowidth{\Width}{$y=\log_{5}x$}\setlength{\Width}{0\Width}%
\settoheight{\Height}{$y=\log_{5}x$}\settodepth{\Depth}{$y=\log_{5}x$}\setlength{\Height}{-0.5\Height}\setlength{\Depth}{0.5\Depth}\addtolength{\Height}{\Depth}%
\put(5.6000000,-0.5000000){\hspace*{\Width}\raisebox{\Height}{$y=\log_{5}x$}}%
%
\settowidth{\Width}{$y=5^x$}\setlength{\Width}{-1\Width}%
\settoheight{\Height}{$y=5^x$}\settodepth{\Depth}{$y=5^x$}\setlength{\Height}{-0.5\Height}\setlength{\Depth}{0.5\Depth}\addtolength{\Height}{\Depth}%
\put(-0.6000000,5.5000000){\hspace*{\Width}\raisebox{\Height}{$y=5^x$}}%
%
\special{pa -1969   -20}\special{pa -1969    20}%
\special{fp}%
\settowidth{\Width}{$-5$}\setlength{\Width}{-0.5\Width}%
\settoheight{\Height}{$-5$}\settodepth{\Depth}{$-5$}\setlength{\Height}{-\Height}%
\put(-5.0000000,-0.1000000){\hspace*{\Width}\raisebox{\Height}{$-5$}}%
%
\special{pa -1575   -20}\special{pa -1575    20}%
\special{fp}%
\settowidth{\Width}{$-4$}\setlength{\Width}{-0.5\Width}%
\settoheight{\Height}{$-4$}\settodepth{\Depth}{$-4$}\setlength{\Height}{-\Height}%
\put(-4.0000000,-0.1000000){\hspace*{\Width}\raisebox{\Height}{$-4$}}%
%
\special{pa -1181   -20}\special{pa -1181    20}%
\special{fp}%
\settowidth{\Width}{$-3$}\setlength{\Width}{-0.5\Width}%
\settoheight{\Height}{$-3$}\settodepth{\Depth}{$-3$}\setlength{\Height}{-\Height}%
\put(-3.0000000,-0.1000000){\hspace*{\Width}\raisebox{\Height}{$-3$}}%
%
\special{pa  -787   -20}\special{pa  -787    20}%
\special{fp}%
\settowidth{\Width}{$-2$}\setlength{\Width}{-0.5\Width}%
\settoheight{\Height}{$-2$}\settodepth{\Depth}{$-2$}\setlength{\Height}{-\Height}%
\put(-2.0000000,-0.1000000){\hspace*{\Width}\raisebox{\Height}{$-2$}}%
%
\special{pa  -394   -20}\special{pa  -394    20}%
\special{fp}%
\settowidth{\Width}{$-1$}\setlength{\Width}{-0.5\Width}%
\settoheight{\Height}{$-1$}\settodepth{\Depth}{$-1$}\setlength{\Height}{-\Height}%
\put(-1.0000000,-0.1000000){\hspace*{\Width}\raisebox{\Height}{$-1$}}%
%
\special{pa   394   -20}\special{pa   394    20}%
\special{fp}%
\settowidth{\Width}{$1$}\setlength{\Width}{-0.5\Width}%
\settoheight{\Height}{$1$}\settodepth{\Depth}{$1$}\setlength{\Height}{-\Height}%
\put(1.0000000,-0.1000000){\hspace*{\Width}\raisebox{\Height}{$1$}}%
%
\special{pa   787   -20}\special{pa   787    20}%
\special{fp}%
\settowidth{\Width}{$2$}\setlength{\Width}{-0.5\Width}%
\settoheight{\Height}{$2$}\settodepth{\Depth}{$2$}\setlength{\Height}{-\Height}%
\put(2.0000000,-0.1000000){\hspace*{\Width}\raisebox{\Height}{$2$}}%
%
\special{pa  1181   -20}\special{pa  1181    20}%
\special{fp}%
\settowidth{\Width}{$3$}\setlength{\Width}{-0.5\Width}%
\settoheight{\Height}{$3$}\settodepth{\Depth}{$3$}\setlength{\Height}{-\Height}%
\put(3.0000000,-0.1000000){\hspace*{\Width}\raisebox{\Height}{$3$}}%
%
\special{pa  1575   -20}\special{pa  1575    20}%
\special{fp}%
\settowidth{\Width}{$4$}\setlength{\Width}{-0.5\Width}%
\settoheight{\Height}{$4$}\settodepth{\Depth}{$4$}\setlength{\Height}{-\Height}%
\put(4.0000000,-0.1000000){\hspace*{\Width}\raisebox{\Height}{$4$}}%
%
\special{pa  1969   -20}\special{pa  1969    20}%
\special{fp}%
\settowidth{\Width}{$5$}\setlength{\Width}{-0.5\Width}%
\settoheight{\Height}{$5$}\settodepth{\Depth}{$5$}\setlength{\Height}{-\Height}%
\put(5.0000000,-0.1000000){\hspace*{\Width}\raisebox{\Height}{$5$}}%
%
\special{pa    20  1969}\special{pa   -20  1969}%
\special{fp}%
\settowidth{\Width}{$-5$}\setlength{\Width}{-1\Width}%
\settoheight{\Height}{$-5$}\settodepth{\Depth}{$-5$}\setlength{\Height}{-0.5\Height}\setlength{\Depth}{0.5\Depth}\addtolength{\Height}{\Depth}%
\put(-0.1000000,-5.0000000){\hspace*{\Width}\raisebox{\Height}{$-5$}}%
%
\special{pa    20  1575}\special{pa   -20  1575}%
\special{fp}%
\settowidth{\Width}{$-4$}\setlength{\Width}{-1\Width}%
\settoheight{\Height}{$-4$}\settodepth{\Depth}{$-4$}\setlength{\Height}{-0.5\Height}\setlength{\Depth}{0.5\Depth}\addtolength{\Height}{\Depth}%
\put(-0.1000000,-4.0000000){\hspace*{\Width}\raisebox{\Height}{$-4$}}%
%
\special{pa    20  1181}\special{pa   -20  1181}%
\special{fp}%
\settowidth{\Width}{$-3$}\setlength{\Width}{-1\Width}%
\settoheight{\Height}{$-3$}\settodepth{\Depth}{$-3$}\setlength{\Height}{-0.5\Height}\setlength{\Depth}{0.5\Depth}\addtolength{\Height}{\Depth}%
\put(-0.1000000,-3.0000000){\hspace*{\Width}\raisebox{\Height}{$-3$}}%
%
\special{pa    20   787}\special{pa   -20   787}%
\special{fp}%
\settowidth{\Width}{$-2$}\setlength{\Width}{-1\Width}%
\settoheight{\Height}{$-2$}\settodepth{\Depth}{$-2$}\setlength{\Height}{-0.5\Height}\setlength{\Depth}{0.5\Depth}\addtolength{\Height}{\Depth}%
\put(-0.1000000,-2.0000000){\hspace*{\Width}\raisebox{\Height}{$-2$}}%
%
\special{pa    20   394}\special{pa   -20   394}%
\special{fp}%
\settowidth{\Width}{$-1$}\setlength{\Width}{-1\Width}%
\settoheight{\Height}{$-1$}\settodepth{\Depth}{$-1$}\setlength{\Height}{-0.5\Height}\setlength{\Depth}{0.5\Depth}\addtolength{\Height}{\Depth}%
\put(-0.1000000,-1.0000000){\hspace*{\Width}\raisebox{\Height}{$-1$}}%
%
\special{pa    20    -0}\special{pa   -20    -0}%
\special{fp}%
\settowidth{\Width}{$0$}\setlength{\Width}{-1\Width}%
\settoheight{\Height}{$0$}\settodepth{\Depth}{$0$}\setlength{\Height}{-0.5\Height}\setlength{\Depth}{0.5\Depth}\addtolength{\Height}{\Depth}%
\put(-0.1000000,0.0000000){\hspace*{\Width}\raisebox{\Height}{$0$}}%
%
\special{pa    20  -394}\special{pa   -20  -394}%
\special{fp}%
\settowidth{\Width}{$1$}\setlength{\Width}{-1\Width}%
\settoheight{\Height}{$1$}\settodepth{\Depth}{$1$}\setlength{\Height}{-0.5\Height}\setlength{\Depth}{0.5\Depth}\addtolength{\Height}{\Depth}%
\put(-0.1000000,1.0000000){\hspace*{\Width}\raisebox{\Height}{$1$}}%
%
\special{pa    20  -787}\special{pa   -20  -787}%
\special{fp}%
\settowidth{\Width}{$2$}\setlength{\Width}{-1\Width}%
\settoheight{\Height}{$2$}\settodepth{\Depth}{$2$}\setlength{\Height}{-0.5\Height}\setlength{\Depth}{0.5\Depth}\addtolength{\Height}{\Depth}%
\put(-0.1000000,2.0000000){\hspace*{\Width}\raisebox{\Height}{$2$}}%
%
\special{pa    20 -1181}\special{pa   -20 -1181}%
\special{fp}%
\settowidth{\Width}{$3$}\setlength{\Width}{-1\Width}%
\settoheight{\Height}{$3$}\settodepth{\Depth}{$3$}\setlength{\Height}{-0.5\Height}\setlength{\Depth}{0.5\Depth}\addtolength{\Height}{\Depth}%
\put(-0.1000000,3.0000000){\hspace*{\Width}\raisebox{\Height}{$3$}}%
%
\special{pa    20 -1575}\special{pa   -20 -1575}%
\special{fp}%
\settowidth{\Width}{$4$}\setlength{\Width}{-1\Width}%
\settoheight{\Height}{$4$}\settodepth{\Depth}{$4$}\setlength{\Height}{-0.5\Height}\setlength{\Depth}{0.5\Depth}\addtolength{\Height}{\Depth}%
\put(-0.1000000,4.0000000){\hspace*{\Width}\raisebox{\Height}{$4$}}%
%
\special{pa    20 -1969}\special{pa   -20 -1969}%
\special{fp}%
\settowidth{\Width}{$5$}\setlength{\Width}{-1\Width}%
\settoheight{\Height}{$5$}\settodepth{\Depth}{$5$}\setlength{\Height}{-0.5\Height}\setlength{\Depth}{0.5\Depth}\addtolength{\Height}{\Depth}%
\put(-0.1000000,5.0000000){\hspace*{\Width}\raisebox{\Height}{$5$}}%
%
\special{pa -2362    -0}\special{pa  2362    -0}%
\special{fp}%
\special{pa     0  2362}\special{pa     0 -2362}%
\special{fp}%
\settowidth{\Width}{$x$}\setlength{\Width}{0\Width}%
\settoheight{\Height}{$x$}\settodepth{\Depth}{$x$}\setlength{\Height}{-0.5\Height}\setlength{\Depth}{0.5\Depth}\addtolength{\Height}{\Depth}%
\put(6.0500000,0.0000000){\hspace*{\Width}\raisebox{\Height}{$x$}}%
%
\settowidth{\Width}{$y$}\setlength{\Width}{-0.5\Width}%
\settoheight{\Height}{$y$}\settodepth{\Depth}{$y$}\setlength{\Height}{\Depth}%
\put(0.0000000,6.0500000){\hspace*{\Width}\raisebox{\Height}{$y$}}%
%
\settowidth{\Width}{O}\setlength{\Width}{-1\Width}%
\settoheight{\Height}{O}\settodepth{\Depth}{O}\setlength{\Height}{-\Height}%
\put(-0.0500000,-0.0500000){\hspace*{\Width}\raisebox{\Height}{O}}%
%
\end{picture}}%}}
\end{layer}

\begin{itemize}
\item
{\normalsize\url{https://s-takato.github.io/dntnet/taisuugraph/taisuu.html}}
\item
$y=a^x$と$y=\log_a x$は\\$y=x$に関して対称
\item
このような関数を\\{\color{red}逆関数}という
\item
対数関数と指数関数は\\
\hspace*{2zw}逆関数どうし
\end{itemize}

\sameslide

\vspace*{18mm}

\slidepage

\begin{layer}{120}{0}
\putnotese{63}{17}{\scalebox{0.5}{%%% /polytech.git/n106/fig/taisuu10.tex 
%%% Generator=taisuujs.cdy 
{\unitlength=1cm%
\begin{picture}%
(12,12)(-6,-6)%
\special{pn 8}%
%
\special{pn 4}%
\special{pa -1969  1969}\special{pa -1969 -1969}%
\special{fp}%
\special{pn 8}%
\special{pn 4}%
\special{pa -1575  1969}\special{pa -1575 -1969}%
\special{fp}%
\special{pn 8}%
\special{pn 4}%
\special{pa -1181  1969}\special{pa -1181 -1969}%
\special{fp}%
\special{pn 8}%
\special{pn 4}%
\special{pa  -787  1969}\special{pa  -787 -1969}%
\special{fp}%
\special{pn 8}%
\special{pn 4}%
\special{pa  -394  1969}\special{pa  -394 -1969}%
\special{fp}%
\special{pn 8}%
\special{pn 4}%
\special{pa     0  1969}\special{pa     0 -1969}%
\special{fp}%
\special{pn 8}%
\special{pn 4}%
\special{pa   394  1969}\special{pa   394 -1969}%
\special{fp}%
\special{pn 8}%
\special{pn 4}%
\special{pa   787  1969}\special{pa   787 -1969}%
\special{fp}%
\special{pn 8}%
\special{pn 4}%
\special{pa  1181  1969}\special{pa  1181 -1969}%
\special{fp}%
\special{pn 8}%
\special{pn 4}%
\special{pa  1575  1969}\special{pa  1575 -1969}%
\special{fp}%
\special{pn 8}%
\special{pn 4}%
\special{pa  1969  1969}\special{pa  1969 -1969}%
\special{fp}%
\special{pn 8}%
\special{pn 4}%
\special{pa -1969  1969}\special{pa  1969  1969}%
\special{fp}%
\special{pn 8}%
\special{pn 4}%
\special{pa -1969  1575}\special{pa  1969  1575}%
\special{fp}%
\special{pn 8}%
\special{pn 4}%
\special{pa -1969  1181}\special{pa  1969  1181}%
\special{fp}%
\special{pn 8}%
\special{pn 4}%
\special{pa -1969   787}\special{pa  1969   787}%
\special{fp}%
\special{pn 8}%
\special{pn 4}%
\special{pa -1969   394}\special{pa  1969   394}%
\special{fp}%
\special{pn 8}%
\special{pn 4}%
\special{pa -1969    -0}\special{pa  1969    -0}%
\special{fp}%
\special{pn 8}%
\special{pn 4}%
\special{pa -1969  -394}\special{pa  1969  -394}%
\special{fp}%
\special{pn 8}%
\special{pn 4}%
\special{pa -1969  -787}\special{pa  1969  -787}%
\special{fp}%
\special{pn 8}%
\special{pn 4}%
\special{pa -1969 -1181}\special{pa  1969 -1181}%
\special{fp}%
\special{pn 8}%
\special{pn 4}%
\special{pa -1969 -1575}\special{pa  1969 -1575}%
\special{fp}%
\special{pn 8}%
\special{pn 4}%
\special{pa -1969 -1969}\special{pa  1969 -1969}%
\special{fp}%
\special{pn 8}%
\special{pa -2362 2362}\special{pa -2335 2335}\special{fp}\special{pa -2307 2307}\special{pa -2279 2279}\special{fp}%
\special{pa -2252 2252}\special{pa -2224 2224}\special{fp}\special{pa -2196 2196}\special{pa -2169 2169}\special{fp}%
\special{pa -2141 2141}\special{pa -2114 2114}\special{fp}\special{pa -2086 2086}\special{pa -2058 2058}\special{fp}%
\special{pa -2031 2031}\special{pa -2003 2003}\special{fp}\special{pa -1975 1975}\special{pa -1948 1948}\special{fp}%
\special{pa -1920 1920}\special{pa -1893 1893}\special{fp}\special{pa -1865 1865}\special{pa -1837 1837}\special{fp}%
\special{pa -1810 1810}\special{pa -1782 1782}\special{fp}\special{pa -1754 1754}\special{pa -1727 1727}\special{fp}%
\special{pa -1699 1699}\special{pa -1672 1672}\special{fp}\special{pa -1644 1644}\special{pa -1616 1616}\special{fp}%
\special{pa -1589 1589}\special{pa -1561 1561}\special{fp}\special{pa -1533 1533}\special{pa -1506 1506}\special{fp}%
\special{pa -1478 1478}\special{pa -1450 1450}\special{fp}\special{pa -1423 1423}\special{pa -1395 1395}\special{fp}%
\special{pa -1368 1368}\special{pa -1340 1340}\special{fp}\special{pa -1312 1312}\special{pa -1285 1285}\special{fp}%
\special{pa -1257 1257}\special{pa -1229 1229}\special{fp}\special{pa -1202 1202}\special{pa -1174 1174}\special{fp}%
\special{pa -1147 1147}\special{pa -1119 1119}\special{fp}\special{pa -1091 1091}\special{pa -1064 1064}\special{fp}%
\special{pa -1036 1036}\special{pa -1008 1008}\special{fp}\special{pa -981 981}\special{pa -953 953}\special{fp}%
\special{pa -926 926}\special{pa -898 898}\special{fp}\special{pa -870 870}\special{pa -843 843}\special{fp}%
\special{pa -815 815}\special{pa -787 787}\special{fp}\special{pa -760 760}\special{pa -732 732}\special{fp}%
\special{pa -705 705}\special{pa -677 677}\special{fp}\special{pa -649 649}\special{pa -622 622}\special{fp}%
\special{pa -594 594}\special{pa -566 566}\special{fp}\special{pa -539 539}\special{pa -511 511}\special{fp}%
\special{pa -483 483}\special{pa -456 456}\special{fp}\special{pa -428 428}\special{pa -401 401}\special{fp}%
\special{pa -373 373}\special{pa -345 345}\special{fp}\special{pa -318 318}\special{pa -290 290}\special{fp}%
\special{pa -262 262}\special{pa -235 235}\special{fp}\special{pa -207 207}\special{pa -180 180}\special{fp}%
\special{pa -152 152}\special{pa -124 124}\special{fp}\special{pa -97 97}\special{pa -69 69}\special{fp}%
\special{pa -41 41}\special{pa -14 14}\special{fp}\special{pa 14 -14}\special{pa 41 -41}\special{fp}%
\special{pa 69 -69}\special{pa 97 -97}\special{fp}\special{pa 124 -124}\special{pa 152 -152}\special{fp}%
\special{pa 180 -180}\special{pa 207 -207}\special{fp}\special{pa 235 -235}\special{pa 262 -262}\special{fp}%
\special{pa 290 -290}\special{pa 318 -318}\special{fp}\special{pa 345 -345}\special{pa 373 -373}\special{fp}%
\special{pa 401 -401}\special{pa 428 -428}\special{fp}\special{pa 456 -456}\special{pa 483 -483}\special{fp}%
\special{pa 511 -511}\special{pa 539 -539}\special{fp}\special{pa 566 -566}\special{pa 594 -594}\special{fp}%
\special{pa 622 -622}\special{pa 649 -649}\special{fp}\special{pa 677 -677}\special{pa 705 -705}\special{fp}%
\special{pa 732 -732}\special{pa 760 -760}\special{fp}\special{pa 787 -787}\special{pa 815 -815}\special{fp}%
\special{pa 843 -843}\special{pa 870 -870}\special{fp}\special{pa 898 -898}\special{pa 926 -926}\special{fp}%
\special{pa 953 -953}\special{pa 981 -981}\special{fp}\special{pa 1008 -1008}\special{pa 1036 -1036}\special{fp}%
\special{pa 1064 -1064}\special{pa 1091 -1091}\special{fp}\special{pa 1119 -1119}\special{pa 1147 -1147}\special{fp}%
\special{pa 1174 -1174}\special{pa 1202 -1202}\special{fp}\special{pa 1229 -1229}\special{pa 1257 -1257}\special{fp}%
\special{pa 1285 -1285}\special{pa 1312 -1312}\special{fp}\special{pa 1340 -1340}\special{pa 1368 -1368}\special{fp}%
\special{pa 1395 -1395}\special{pa 1423 -1423}\special{fp}\special{pa 1450 -1450}\special{pa 1478 -1478}\special{fp}%
\special{pa 1506 -1506}\special{pa 1533 -1533}\special{fp}\special{pa 1561 -1561}\special{pa 1589 -1589}\special{fp}%
\special{pa 1616 -1616}\special{pa 1644 -1644}\special{fp}\special{pa 1672 -1672}\special{pa 1699 -1699}\special{fp}%
\special{pa 1727 -1727}\special{pa 1754 -1754}\special{fp}\special{pa 1782 -1782}\special{pa 1810 -1810}\special{fp}%
\special{pa 1837 -1837}\special{pa 1865 -1865}\special{fp}\special{pa 1893 -1893}\special{pa 1920 -1920}\special{fp}%
\special{pa 1948 -1948}\special{pa 1975 -1975}\special{fp}\special{pa 2003 -2003}\special{pa 2031 -2031}\special{fp}%
\special{pa 2058 -2058}\special{pa 2086 -2086}\special{fp}\special{pa 2114 -2114}\special{pa 2141 -2141}\special{fp}%
\special{pa 2169 -2169}\special{pa 2196 -2196}\special{fp}\special{pa 2224 -2224}\special{pa 2252 -2252}\special{fp}%
\special{pa 2279 -2279}\special{pa 2307 -2307}\special{fp}\special{pa 2335 -2335}\special{pa 2362 -2362}\special{fp}%
%
%
{%
\color[cmyk]{1,0,0,0}%
\special{pn 12}%
\special{pa -2362    -0}\special{pa -2315    -0}\special{pa -2268    -0}\special{pa -2220    -0}%
\special{pa -2173    -0}\special{pa -2126    -0}\special{pa -2079    -0}\special{pa -2031    -0}%
\special{pa -1984    -0}\special{pa -1937    -0}\special{pa -1890    -0}\special{pa -1843    -0}%
\special{pa -1795    -0}\special{pa -1748    -0}\special{pa -1701    -0}\special{pa -1654    -0}%
\special{pa -1606    -0}\special{pa -1559    -0}\special{pa -1512    -0}\special{pa -1465    -0}%
\special{pa -1417    -0}\special{pa -1370    -0}\special{pa -1323    -0}\special{pa -1276    -0}%
\special{pa -1228    -0}\special{pa -1181    -0}\special{pa -1134    -1}\special{pa -1087    -1}%
\special{pa -1039    -1}\special{pa  -992    -1}\special{pa  -945    -2}\special{pa  -898    -2}%
\special{pa  -850    -3}\special{pa  -803    -4}\special{pa  -756    -5}\special{pa  -709    -6}%
\special{pa  -661    -8}\special{pa  -614   -11}\special{pa  -567   -14}\special{pa  -520   -19}%
\special{pa  -472   -25}\special{pa  -425   -33}\special{pa  -378   -43}\special{pa  -331   -57}%
\special{pa  -283   -75}\special{pa  -236   -99}\special{pa  -189  -130}\special{pa  -142  -172}%
\special{pa   -94  -227}\special{pa   -47  -299}\special{pa     0  -394}\special{pa    47  -519}%
\special{pa    94  -684}\special{pa   142  -902}\special{pa   189 -1189}\special{pa   236 -1567}%
\special{pa   283 -2066}\special{pa   305 -2362}%
\special{fp}%
\special{pn 8}%
}%
\special{pn 12}%
\special{pa     0  2362}\special{pa     0  2315}\special{pa     0  2268}\special{pa     0  2220}%
\special{pa     0  2173}\special{pa     0  2126}\special{pa     0  2079}\special{pa     0  2031}%
\special{pa     0  1984}\special{pa     0  1937}\special{pa     0  1890}\special{pa     0  1843}%
\special{pa     0  1795}\special{pa     0  1748}\special{pa     0  1701}\special{pa     0  1654}%
\special{pa     0  1606}\special{pa     0  1559}\special{pa     0  1512}\special{pa     0  1465}%
\special{pa     0  1417}\special{pa     0  1370}\special{pa     0  1323}\special{pa     0  1276}%
\special{pa     0  1228}\special{pa     0  1181}\special{pa     1  1134}\special{pa     1  1087}%
\special{pa     1  1039}\special{pa     1   992}\special{pa     2   945}\special{pa     2   898}%
\special{pa     3   850}\special{pa     4   803}\special{pa     5   756}\special{pa     6   709}%
\special{pa     8   661}\special{pa    11   614}\special{pa    14   567}\special{pa    19   520}%
\special{pa    25   472}\special{pa    33   425}\special{pa    43   378}\special{pa    57   331}%
\special{pa    75   283}\special{pa    99   236}\special{pa   130   189}\special{pa   172   142}%
\special{pa   227    94}\special{pa   299    47}\special{pa   394    -0}\special{pa   519   -47}%
\special{pa   684   -94}\special{pa   902  -142}\special{pa  1189  -189}\special{pa  1567  -236}%
\special{pa  2066  -283}\special{pa  2362  -305}%
\special{fp}%
\special{pn 8}%
\settowidth{\Width}{$y=\log_{10}x$}\setlength{\Width}{0\Width}%
\settoheight{\Height}{$y=\log_{10}x$}\settodepth{\Depth}{$y=\log_{10}x$}\setlength{\Height}{-0.5\Height}\setlength{\Depth}{0.5\Depth}\addtolength{\Height}{\Depth}%
\put(5.6000000,-0.5000000){\hspace*{\Width}\raisebox{\Height}{$y=\log_{10}x$}}%
%
\settowidth{\Width}{$y=10^x$}\setlength{\Width}{-1\Width}%
\settoheight{\Height}{$y=10^x$}\settodepth{\Depth}{$y=10^x$}\setlength{\Height}{-0.5\Height}\setlength{\Depth}{0.5\Depth}\addtolength{\Height}{\Depth}%
\put(-0.6000000,5.5000000){\hspace*{\Width}\raisebox{\Height}{$y=10^x$}}%
%
\special{pa -1969   -20}\special{pa -1969    20}%
\special{fp}%
\settowidth{\Width}{$-5$}\setlength{\Width}{-0.5\Width}%
\settoheight{\Height}{$-5$}\settodepth{\Depth}{$-5$}\setlength{\Height}{-\Height}%
\put(-5.0000000,-0.1000000){\hspace*{\Width}\raisebox{\Height}{$-5$}}%
%
\special{pa -1575   -20}\special{pa -1575    20}%
\special{fp}%
\settowidth{\Width}{$-4$}\setlength{\Width}{-0.5\Width}%
\settoheight{\Height}{$-4$}\settodepth{\Depth}{$-4$}\setlength{\Height}{-\Height}%
\put(-4.0000000,-0.1000000){\hspace*{\Width}\raisebox{\Height}{$-4$}}%
%
\special{pa -1181   -20}\special{pa -1181    20}%
\special{fp}%
\settowidth{\Width}{$-3$}\setlength{\Width}{-0.5\Width}%
\settoheight{\Height}{$-3$}\settodepth{\Depth}{$-3$}\setlength{\Height}{-\Height}%
\put(-3.0000000,-0.1000000){\hspace*{\Width}\raisebox{\Height}{$-3$}}%
%
\special{pa  -787   -20}\special{pa  -787    20}%
\special{fp}%
\settowidth{\Width}{$-2$}\setlength{\Width}{-0.5\Width}%
\settoheight{\Height}{$-2$}\settodepth{\Depth}{$-2$}\setlength{\Height}{-\Height}%
\put(-2.0000000,-0.1000000){\hspace*{\Width}\raisebox{\Height}{$-2$}}%
%
\special{pa  -394   -20}\special{pa  -394    20}%
\special{fp}%
\settowidth{\Width}{$-1$}\setlength{\Width}{-0.5\Width}%
\settoheight{\Height}{$-1$}\settodepth{\Depth}{$-1$}\setlength{\Height}{-\Height}%
\put(-1.0000000,-0.1000000){\hspace*{\Width}\raisebox{\Height}{$-1$}}%
%
\special{pa   394   -20}\special{pa   394    20}%
\special{fp}%
\settowidth{\Width}{$1$}\setlength{\Width}{-0.5\Width}%
\settoheight{\Height}{$1$}\settodepth{\Depth}{$1$}\setlength{\Height}{-\Height}%
\put(1.0000000,-0.1000000){\hspace*{\Width}\raisebox{\Height}{$1$}}%
%
\special{pa   787   -20}\special{pa   787    20}%
\special{fp}%
\settowidth{\Width}{$2$}\setlength{\Width}{-0.5\Width}%
\settoheight{\Height}{$2$}\settodepth{\Depth}{$2$}\setlength{\Height}{-\Height}%
\put(2.0000000,-0.1000000){\hspace*{\Width}\raisebox{\Height}{$2$}}%
%
\special{pa  1181   -20}\special{pa  1181    20}%
\special{fp}%
\settowidth{\Width}{$3$}\setlength{\Width}{-0.5\Width}%
\settoheight{\Height}{$3$}\settodepth{\Depth}{$3$}\setlength{\Height}{-\Height}%
\put(3.0000000,-0.1000000){\hspace*{\Width}\raisebox{\Height}{$3$}}%
%
\special{pa  1575   -20}\special{pa  1575    20}%
\special{fp}%
\settowidth{\Width}{$4$}\setlength{\Width}{-0.5\Width}%
\settoheight{\Height}{$4$}\settodepth{\Depth}{$4$}\setlength{\Height}{-\Height}%
\put(4.0000000,-0.1000000){\hspace*{\Width}\raisebox{\Height}{$4$}}%
%
\special{pa  1969   -20}\special{pa  1969    20}%
\special{fp}%
\settowidth{\Width}{$5$}\setlength{\Width}{-0.5\Width}%
\settoheight{\Height}{$5$}\settodepth{\Depth}{$5$}\setlength{\Height}{-\Height}%
\put(5.0000000,-0.1000000){\hspace*{\Width}\raisebox{\Height}{$5$}}%
%
\special{pa    20  1969}\special{pa   -20  1969}%
\special{fp}%
\settowidth{\Width}{$-5$}\setlength{\Width}{-1\Width}%
\settoheight{\Height}{$-5$}\settodepth{\Depth}{$-5$}\setlength{\Height}{-0.5\Height}\setlength{\Depth}{0.5\Depth}\addtolength{\Height}{\Depth}%
\put(-0.1000000,-5.0000000){\hspace*{\Width}\raisebox{\Height}{$-5$}}%
%
\special{pa    20  1575}\special{pa   -20  1575}%
\special{fp}%
\settowidth{\Width}{$-4$}\setlength{\Width}{-1\Width}%
\settoheight{\Height}{$-4$}\settodepth{\Depth}{$-4$}\setlength{\Height}{-0.5\Height}\setlength{\Depth}{0.5\Depth}\addtolength{\Height}{\Depth}%
\put(-0.1000000,-4.0000000){\hspace*{\Width}\raisebox{\Height}{$-4$}}%
%
\special{pa    20  1181}\special{pa   -20  1181}%
\special{fp}%
\settowidth{\Width}{$-3$}\setlength{\Width}{-1\Width}%
\settoheight{\Height}{$-3$}\settodepth{\Depth}{$-3$}\setlength{\Height}{-0.5\Height}\setlength{\Depth}{0.5\Depth}\addtolength{\Height}{\Depth}%
\put(-0.1000000,-3.0000000){\hspace*{\Width}\raisebox{\Height}{$-3$}}%
%
\special{pa    20   787}\special{pa   -20   787}%
\special{fp}%
\settowidth{\Width}{$-2$}\setlength{\Width}{-1\Width}%
\settoheight{\Height}{$-2$}\settodepth{\Depth}{$-2$}\setlength{\Height}{-0.5\Height}\setlength{\Depth}{0.5\Depth}\addtolength{\Height}{\Depth}%
\put(-0.1000000,-2.0000000){\hspace*{\Width}\raisebox{\Height}{$-2$}}%
%
\special{pa    20   394}\special{pa   -20   394}%
\special{fp}%
\settowidth{\Width}{$-1$}\setlength{\Width}{-1\Width}%
\settoheight{\Height}{$-1$}\settodepth{\Depth}{$-1$}\setlength{\Height}{-0.5\Height}\setlength{\Depth}{0.5\Depth}\addtolength{\Height}{\Depth}%
\put(-0.1000000,-1.0000000){\hspace*{\Width}\raisebox{\Height}{$-1$}}%
%
\special{pa    20    -0}\special{pa   -20    -0}%
\special{fp}%
\settowidth{\Width}{$0$}\setlength{\Width}{-1\Width}%
\settoheight{\Height}{$0$}\settodepth{\Depth}{$0$}\setlength{\Height}{-0.5\Height}\setlength{\Depth}{0.5\Depth}\addtolength{\Height}{\Depth}%
\put(-0.1000000,0.0000000){\hspace*{\Width}\raisebox{\Height}{$0$}}%
%
\special{pa    20  -394}\special{pa   -20  -394}%
\special{fp}%
\settowidth{\Width}{$1$}\setlength{\Width}{-1\Width}%
\settoheight{\Height}{$1$}\settodepth{\Depth}{$1$}\setlength{\Height}{-0.5\Height}\setlength{\Depth}{0.5\Depth}\addtolength{\Height}{\Depth}%
\put(-0.1000000,1.0000000){\hspace*{\Width}\raisebox{\Height}{$1$}}%
%
\special{pa    20  -787}\special{pa   -20  -787}%
\special{fp}%
\settowidth{\Width}{$2$}\setlength{\Width}{-1\Width}%
\settoheight{\Height}{$2$}\settodepth{\Depth}{$2$}\setlength{\Height}{-0.5\Height}\setlength{\Depth}{0.5\Depth}\addtolength{\Height}{\Depth}%
\put(-0.1000000,2.0000000){\hspace*{\Width}\raisebox{\Height}{$2$}}%
%
\special{pa    20 -1181}\special{pa   -20 -1181}%
\special{fp}%
\settowidth{\Width}{$3$}\setlength{\Width}{-1\Width}%
\settoheight{\Height}{$3$}\settodepth{\Depth}{$3$}\setlength{\Height}{-0.5\Height}\setlength{\Depth}{0.5\Depth}\addtolength{\Height}{\Depth}%
\put(-0.1000000,3.0000000){\hspace*{\Width}\raisebox{\Height}{$3$}}%
%
\special{pa    20 -1575}\special{pa   -20 -1575}%
\special{fp}%
\settowidth{\Width}{$4$}\setlength{\Width}{-1\Width}%
\settoheight{\Height}{$4$}\settodepth{\Depth}{$4$}\setlength{\Height}{-0.5\Height}\setlength{\Depth}{0.5\Depth}\addtolength{\Height}{\Depth}%
\put(-0.1000000,4.0000000){\hspace*{\Width}\raisebox{\Height}{$4$}}%
%
\special{pa    20 -1969}\special{pa   -20 -1969}%
\special{fp}%
\settowidth{\Width}{$5$}\setlength{\Width}{-1\Width}%
\settoheight{\Height}{$5$}\settodepth{\Depth}{$5$}\setlength{\Height}{-0.5\Height}\setlength{\Depth}{0.5\Depth}\addtolength{\Height}{\Depth}%
\put(-0.1000000,5.0000000){\hspace*{\Width}\raisebox{\Height}{$5$}}%
%
\special{pa -2362    -0}\special{pa  2362    -0}%
\special{fp}%
\special{pa     0  2362}\special{pa     0 -2362}%
\special{fp}%
\settowidth{\Width}{$x$}\setlength{\Width}{0\Width}%
\settoheight{\Height}{$x$}\settodepth{\Depth}{$x$}\setlength{\Height}{-0.5\Height}\setlength{\Depth}{0.5\Depth}\addtolength{\Height}{\Depth}%
\put(6.0500000,0.0000000){\hspace*{\Width}\raisebox{\Height}{$x$}}%
%
\settowidth{\Width}{$y$}\setlength{\Width}{-0.5\Width}%
\settoheight{\Height}{$y$}\settodepth{\Depth}{$y$}\setlength{\Height}{\Depth}%
\put(0.0000000,6.0500000){\hspace*{\Width}\raisebox{\Height}{$y$}}%
%
\settowidth{\Width}{O}\setlength{\Width}{-1\Width}%
\settoheight{\Height}{O}\settodepth{\Depth}{O}\setlength{\Height}{-\Height}%
\put(-0.0500000,-0.0500000){\hspace*{\Width}\raisebox{\Height}{O}}%
%
\end{picture}}%}}
\end{layer}

\begin{itemize}
\item
{\normalsize\url{https://s-takato.github.io/dntnet/taisuugraph/taisuu.html}}
\item
$y=a^x$と$y=\log_a x$は\\$y=x$に関して対称
\item
このような関数を\\{\color{red}逆関数}という
\item
対数関数と指数関数は\\
\hspace*{2zw}逆関数どうし
\end{itemize}

\sameslide

\vspace*{18mm}

\slidepage

\begin{layer}{120}{0}
\putnotese{63}{17}{\scalebox{0.5}{%%% /polytech.git/n106/fig/taisuum2.tex 
%%% Generator=taisuujs.cdy 
{\unitlength=1cm%
\begin{picture}%
(12,12)(-6,-6)%
\special{pn 8}%
%
\special{pn 4}%
\special{pa -1969  1969}\special{pa -1969 -1969}%
\special{fp}%
\special{pn 8}%
\special{pn 4}%
\special{pa -1575  1969}\special{pa -1575 -1969}%
\special{fp}%
\special{pn 8}%
\special{pn 4}%
\special{pa -1181  1969}\special{pa -1181 -1969}%
\special{fp}%
\special{pn 8}%
\special{pn 4}%
\special{pa  -787  1969}\special{pa  -787 -1969}%
\special{fp}%
\special{pn 8}%
\special{pn 4}%
\special{pa  -394  1969}\special{pa  -394 -1969}%
\special{fp}%
\special{pn 8}%
\special{pn 4}%
\special{pa     0  1969}\special{pa     0 -1969}%
\special{fp}%
\special{pn 8}%
\special{pn 4}%
\special{pa   394  1969}\special{pa   394 -1969}%
\special{fp}%
\special{pn 8}%
\special{pn 4}%
\special{pa   787  1969}\special{pa   787 -1969}%
\special{fp}%
\special{pn 8}%
\special{pn 4}%
\special{pa  1181  1969}\special{pa  1181 -1969}%
\special{fp}%
\special{pn 8}%
\special{pn 4}%
\special{pa  1575  1969}\special{pa  1575 -1969}%
\special{fp}%
\special{pn 8}%
\special{pn 4}%
\special{pa  1969  1969}\special{pa  1969 -1969}%
\special{fp}%
\special{pn 8}%
\special{pn 4}%
\special{pa -1969  1969}\special{pa  1969  1969}%
\special{fp}%
\special{pn 8}%
\special{pn 4}%
\special{pa -1969  1575}\special{pa  1969  1575}%
\special{fp}%
\special{pn 8}%
\special{pn 4}%
\special{pa -1969  1181}\special{pa  1969  1181}%
\special{fp}%
\special{pn 8}%
\special{pn 4}%
\special{pa -1969   787}\special{pa  1969   787}%
\special{fp}%
\special{pn 8}%
\special{pn 4}%
\special{pa -1969   394}\special{pa  1969   394}%
\special{fp}%
\special{pn 8}%
\special{pn 4}%
\special{pa -1969    -0}\special{pa  1969    -0}%
\special{fp}%
\special{pn 8}%
\special{pn 4}%
\special{pa -1969  -394}\special{pa  1969  -394}%
\special{fp}%
\special{pn 8}%
\special{pn 4}%
\special{pa -1969  -787}\special{pa  1969  -787}%
\special{fp}%
\special{pn 8}%
\special{pn 4}%
\special{pa -1969 -1181}\special{pa  1969 -1181}%
\special{fp}%
\special{pn 8}%
\special{pn 4}%
\special{pa -1969 -1575}\special{pa  1969 -1575}%
\special{fp}%
\special{pn 8}%
\special{pn 4}%
\special{pa -1969 -1969}\special{pa  1969 -1969}%
\special{fp}%
\special{pn 8}%
\special{pa -2362 2362}\special{pa -2335 2335}\special{fp}\special{pa -2307 2307}\special{pa -2279 2279}\special{fp}%
\special{pa -2252 2252}\special{pa -2224 2224}\special{fp}\special{pa -2196 2196}\special{pa -2169 2169}\special{fp}%
\special{pa -2141 2141}\special{pa -2114 2114}\special{fp}\special{pa -2086 2086}\special{pa -2058 2058}\special{fp}%
\special{pa -2031 2031}\special{pa -2003 2003}\special{fp}\special{pa -1975 1975}\special{pa -1948 1948}\special{fp}%
\special{pa -1920 1920}\special{pa -1893 1893}\special{fp}\special{pa -1865 1865}\special{pa -1837 1837}\special{fp}%
\special{pa -1810 1810}\special{pa -1782 1782}\special{fp}\special{pa -1754 1754}\special{pa -1727 1727}\special{fp}%
\special{pa -1699 1699}\special{pa -1672 1672}\special{fp}\special{pa -1644 1644}\special{pa -1616 1616}\special{fp}%
\special{pa -1589 1589}\special{pa -1561 1561}\special{fp}\special{pa -1533 1533}\special{pa -1506 1506}\special{fp}%
\special{pa -1478 1478}\special{pa -1450 1450}\special{fp}\special{pa -1423 1423}\special{pa -1395 1395}\special{fp}%
\special{pa -1368 1368}\special{pa -1340 1340}\special{fp}\special{pa -1312 1312}\special{pa -1285 1285}\special{fp}%
\special{pa -1257 1257}\special{pa -1229 1229}\special{fp}\special{pa -1202 1202}\special{pa -1174 1174}\special{fp}%
\special{pa -1147 1147}\special{pa -1119 1119}\special{fp}\special{pa -1091 1091}\special{pa -1064 1064}\special{fp}%
\special{pa -1036 1036}\special{pa -1008 1008}\special{fp}\special{pa -981 981}\special{pa -953 953}\special{fp}%
\special{pa -926 926}\special{pa -898 898}\special{fp}\special{pa -870 870}\special{pa -843 843}\special{fp}%
\special{pa -815 815}\special{pa -787 787}\special{fp}\special{pa -760 760}\special{pa -732 732}\special{fp}%
\special{pa -705 705}\special{pa -677 677}\special{fp}\special{pa -649 649}\special{pa -622 622}\special{fp}%
\special{pa -594 594}\special{pa -566 566}\special{fp}\special{pa -539 539}\special{pa -511 511}\special{fp}%
\special{pa -483 483}\special{pa -456 456}\special{fp}\special{pa -428 428}\special{pa -401 401}\special{fp}%
\special{pa -373 373}\special{pa -345 345}\special{fp}\special{pa -318 318}\special{pa -290 290}\special{fp}%
\special{pa -262 262}\special{pa -235 235}\special{fp}\special{pa -207 207}\special{pa -180 180}\special{fp}%
\special{pa -152 152}\special{pa -124 124}\special{fp}\special{pa -97 97}\special{pa -69 69}\special{fp}%
\special{pa -41 41}\special{pa -14 14}\special{fp}\special{pa 14 -14}\special{pa 41 -41}\special{fp}%
\special{pa 69 -69}\special{pa 97 -97}\special{fp}\special{pa 124 -124}\special{pa 152 -152}\special{fp}%
\special{pa 180 -180}\special{pa 207 -207}\special{fp}\special{pa 235 -235}\special{pa 262 -262}\special{fp}%
\special{pa 290 -290}\special{pa 318 -318}\special{fp}\special{pa 345 -345}\special{pa 373 -373}\special{fp}%
\special{pa 401 -401}\special{pa 428 -428}\special{fp}\special{pa 456 -456}\special{pa 483 -483}\special{fp}%
\special{pa 511 -511}\special{pa 539 -539}\special{fp}\special{pa 566 -566}\special{pa 594 -594}\special{fp}%
\special{pa 622 -622}\special{pa 649 -649}\special{fp}\special{pa 677 -677}\special{pa 705 -705}\special{fp}%
\special{pa 732 -732}\special{pa 760 -760}\special{fp}\special{pa 787 -787}\special{pa 815 -815}\special{fp}%
\special{pa 843 -843}\special{pa 870 -870}\special{fp}\special{pa 898 -898}\special{pa 926 -926}\special{fp}%
\special{pa 953 -953}\special{pa 981 -981}\special{fp}\special{pa 1008 -1008}\special{pa 1036 -1036}\special{fp}%
\special{pa 1064 -1064}\special{pa 1091 -1091}\special{fp}\special{pa 1119 -1119}\special{pa 1147 -1147}\special{fp}%
\special{pa 1174 -1174}\special{pa 1202 -1202}\special{fp}\special{pa 1229 -1229}\special{pa 1257 -1257}\special{fp}%
\special{pa 1285 -1285}\special{pa 1312 -1312}\special{fp}\special{pa 1340 -1340}\special{pa 1368 -1368}\special{fp}%
\special{pa 1395 -1395}\special{pa 1423 -1423}\special{fp}\special{pa 1450 -1450}\special{pa 1478 -1478}\special{fp}%
\special{pa 1506 -1506}\special{pa 1533 -1533}\special{fp}\special{pa 1561 -1561}\special{pa 1589 -1589}\special{fp}%
\special{pa 1616 -1616}\special{pa 1644 -1644}\special{fp}\special{pa 1672 -1672}\special{pa 1699 -1699}\special{fp}%
\special{pa 1727 -1727}\special{pa 1754 -1754}\special{fp}\special{pa 1782 -1782}\special{pa 1810 -1810}\special{fp}%
\special{pa 1837 -1837}\special{pa 1865 -1865}\special{fp}\special{pa 1893 -1893}\special{pa 1920 -1920}\special{fp}%
\special{pa 1948 -1948}\special{pa 1975 -1975}\special{fp}\special{pa 2003 -2003}\special{pa 2031 -2031}\special{fp}%
\special{pa 2058 -2058}\special{pa 2086 -2086}\special{fp}\special{pa 2114 -2114}\special{pa 2141 -2141}\special{fp}%
\special{pa 2169 -2169}\special{pa 2196 -2196}\special{fp}\special{pa 2224 -2224}\special{pa 2252 -2252}\special{fp}%
\special{pa 2279 -2279}\special{pa 2307 -2307}\special{fp}\special{pa 2335 -2335}\special{pa 2362 -2362}\special{fp}%
%
%
{%
\color[cmyk]{1,0,0,0}%
\special{pn 12}%
\special{pa -1017 -2362}\special{pa  -992 -2258}\special{pa  -945 -2078}\special{pa  -898 -1912}%
\special{pa  -850 -1760}\special{pa  -803 -1619}\special{pa  -756 -1490}\special{pa  -709 -1371}%
\special{pa  -661 -1262}\special{pa  -614 -1161}\special{pa  -567 -1068}\special{pa  -520  -983}%
\special{pa  -472  -904}\special{pa  -425  -832}\special{pa  -378  -766}\special{pa  -331  -705}%
\special{pa  -283  -648}\special{pa  -236  -597}\special{pa  -189  -549}\special{pa  -142  -505}%
\special{pa   -94  -465}\special{pa   -47  -428}\special{pa     0  -394}\special{pa    47  -362}%
\special{pa    94  -333}\special{pa   142  -307}\special{pa   189  -282}\special{pa   236  -260}%
\special{pa   283  -239}\special{pa   331  -220}\special{pa   378  -202}\special{pa   425  -186}%
\special{pa   472  -171}\special{pa   520  -158}\special{pa   567  -145}\special{pa   614  -134}%
\special{pa   661  -123}\special{pa   709  -113}\special{pa   756  -104}\special{pa   803   -96}%
\special{pa   850   -88}\special{pa   898   -81}\special{pa   945   -75}\special{pa   992   -69}%
\special{pa  1039   -63}\special{pa  1087   -58}\special{pa  1134   -53}\special{pa  1181   -49}%
\special{pa  1228   -45}\special{pa  1276   -42}\special{pa  1323   -38}\special{pa  1370   -35}%
\special{pa  1417   -32}\special{pa  1465   -30}\special{pa  1512   -27}\special{pa  1559   -25}%
\special{pa  1606   -23}\special{pa  1654   -21}\special{pa  1701   -20}\special{pa  1748   -18}%
\special{pa  1795   -17}\special{pa  1843   -15}\special{pa  1890   -14}\special{pa  1937   -13}%
\special{pa  1984   -12}\special{pa  2031   -11}\special{pa  2079   -10}\special{pa  2126    -9}%
\special{pa  2173    -9}\special{pa  2220    -8}\special{pa  2268    -7}\special{pa  2315    -7}%
\special{pa  2362    -6}%
\special{fp}%
\special{pn 8}%
}%
\special{pn 12}%
\special{pa  2362  1017}\special{pa  2258   992}\special{pa  2078   945}\special{pa  1912   898}%
\special{pa  1760   850}\special{pa  1619   803}\special{pa  1490   756}\special{pa  1371   709}%
\special{pa  1262   661}\special{pa  1161   614}\special{pa  1068   567}\special{pa   983   520}%
\special{pa   904   472}\special{pa   832   425}\special{pa   766   378}\special{pa   705   331}%
\special{pa   648   283}\special{pa   597   236}\special{pa   549   189}\special{pa   505   142}%
\special{pa   465    94}\special{pa   428    47}\special{pa   394    -0}\special{pa   362   -47}%
\special{pa   333   -94}\special{pa   307  -142}\special{pa   282  -189}\special{pa   260  -236}%
\special{pa   239  -283}\special{pa   220  -331}\special{pa   202  -378}\special{pa   186  -425}%
\special{pa   171  -472}\special{pa   158  -520}\special{pa   145  -567}\special{pa   134  -614}%
\special{pa   123  -661}\special{pa   113  -709}\special{pa   104  -756}\special{pa    96  -803}%
\special{pa    88  -850}\special{pa    81  -898}\special{pa    75  -945}\special{pa    69  -992}%
\special{pa    63 -1039}\special{pa    58 -1087}\special{pa    53 -1134}\special{pa    49 -1181}%
\special{pa    45 -1228}\special{pa    42 -1276}\special{pa    38 -1323}\special{pa    35 -1370}%
\special{pa    32 -1417}\special{pa    30 -1465}\special{pa    27 -1512}\special{pa    25 -1559}%
\special{pa    23 -1606}\special{pa    21 -1654}\special{pa    20 -1701}\special{pa    18 -1748}%
\special{pa    17 -1795}\special{pa    15 -1843}\special{pa    14 -1890}\special{pa    13 -1937}%
\special{pa    12 -1984}\special{pa    11 -2031}\special{pa    10 -2079}\special{pa     9 -2126}%
\special{pa     9 -2173}\special{pa     8 -2220}\special{pa     7 -2268}\special{pa     7 -2315}%
\special{pa     6 -2362}%
\special{fp}%
\special{pn 8}%
\settowidth{\Width}{$y=\log_{0.5}x$}\setlength{\Width}{0\Width}%
\settoheight{\Height}{$y=\log_{0.5}x$}\settodepth{\Depth}{$y=\log_{0.5}x$}\setlength{\Height}{-0.5\Height}\setlength{\Depth}{0.5\Depth}\addtolength{\Height}{\Depth}%
\put(5.6000000,-0.5000000){\hspace*{\Width}\raisebox{\Height}{$y=\log_{0.5}x$}}%
%
\settowidth{\Width}{$y=0.5^x$}\setlength{\Width}{-1\Width}%
\settoheight{\Height}{$y=0.5^x$}\settodepth{\Depth}{$y=0.5^x$}\setlength{\Height}{-0.5\Height}\setlength{\Depth}{0.5\Depth}\addtolength{\Height}{\Depth}%
\put(-0.6000000,5.5000000){\hspace*{\Width}\raisebox{\Height}{$y=0.5^x$}}%
%
\special{pa -1969   -20}\special{pa -1969    20}%
\special{fp}%
\settowidth{\Width}{$-5$}\setlength{\Width}{-0.5\Width}%
\settoheight{\Height}{$-5$}\settodepth{\Depth}{$-5$}\setlength{\Height}{-\Height}%
\put(-5.0000000,-0.1000000){\hspace*{\Width}\raisebox{\Height}{$-5$}}%
%
\special{pa -1575   -20}\special{pa -1575    20}%
\special{fp}%
\settowidth{\Width}{$-4$}\setlength{\Width}{-0.5\Width}%
\settoheight{\Height}{$-4$}\settodepth{\Depth}{$-4$}\setlength{\Height}{-\Height}%
\put(-4.0000000,-0.1000000){\hspace*{\Width}\raisebox{\Height}{$-4$}}%
%
\special{pa -1181   -20}\special{pa -1181    20}%
\special{fp}%
\settowidth{\Width}{$-3$}\setlength{\Width}{-0.5\Width}%
\settoheight{\Height}{$-3$}\settodepth{\Depth}{$-3$}\setlength{\Height}{-\Height}%
\put(-3.0000000,-0.1000000){\hspace*{\Width}\raisebox{\Height}{$-3$}}%
%
\special{pa  -787   -20}\special{pa  -787    20}%
\special{fp}%
\settowidth{\Width}{$-2$}\setlength{\Width}{-0.5\Width}%
\settoheight{\Height}{$-2$}\settodepth{\Depth}{$-2$}\setlength{\Height}{-\Height}%
\put(-2.0000000,-0.1000000){\hspace*{\Width}\raisebox{\Height}{$-2$}}%
%
\special{pa  -394   -20}\special{pa  -394    20}%
\special{fp}%
\settowidth{\Width}{$-1$}\setlength{\Width}{-0.5\Width}%
\settoheight{\Height}{$-1$}\settodepth{\Depth}{$-1$}\setlength{\Height}{-\Height}%
\put(-1.0000000,-0.1000000){\hspace*{\Width}\raisebox{\Height}{$-1$}}%
%
\special{pa   394   -20}\special{pa   394    20}%
\special{fp}%
\settowidth{\Width}{$1$}\setlength{\Width}{-0.5\Width}%
\settoheight{\Height}{$1$}\settodepth{\Depth}{$1$}\setlength{\Height}{-\Height}%
\put(1.0000000,-0.1000000){\hspace*{\Width}\raisebox{\Height}{$1$}}%
%
\special{pa   787   -20}\special{pa   787    20}%
\special{fp}%
\settowidth{\Width}{$2$}\setlength{\Width}{-0.5\Width}%
\settoheight{\Height}{$2$}\settodepth{\Depth}{$2$}\setlength{\Height}{-\Height}%
\put(2.0000000,-0.1000000){\hspace*{\Width}\raisebox{\Height}{$2$}}%
%
\special{pa  1181   -20}\special{pa  1181    20}%
\special{fp}%
\settowidth{\Width}{$3$}\setlength{\Width}{-0.5\Width}%
\settoheight{\Height}{$3$}\settodepth{\Depth}{$3$}\setlength{\Height}{-\Height}%
\put(3.0000000,-0.1000000){\hspace*{\Width}\raisebox{\Height}{$3$}}%
%
\special{pa  1575   -20}\special{pa  1575    20}%
\special{fp}%
\settowidth{\Width}{$4$}\setlength{\Width}{-0.5\Width}%
\settoheight{\Height}{$4$}\settodepth{\Depth}{$4$}\setlength{\Height}{-\Height}%
\put(4.0000000,-0.1000000){\hspace*{\Width}\raisebox{\Height}{$4$}}%
%
\special{pa  1969   -20}\special{pa  1969    20}%
\special{fp}%
\settowidth{\Width}{$5$}\setlength{\Width}{-0.5\Width}%
\settoheight{\Height}{$5$}\settodepth{\Depth}{$5$}\setlength{\Height}{-\Height}%
\put(5.0000000,-0.1000000){\hspace*{\Width}\raisebox{\Height}{$5$}}%
%
\special{pa    20  1969}\special{pa   -20  1969}%
\special{fp}%
\settowidth{\Width}{$-5$}\setlength{\Width}{-1\Width}%
\settoheight{\Height}{$-5$}\settodepth{\Depth}{$-5$}\setlength{\Height}{-0.5\Height}\setlength{\Depth}{0.5\Depth}\addtolength{\Height}{\Depth}%
\put(-0.1000000,-5.0000000){\hspace*{\Width}\raisebox{\Height}{$-5$}}%
%
\special{pa    20  1575}\special{pa   -20  1575}%
\special{fp}%
\settowidth{\Width}{$-4$}\setlength{\Width}{-1\Width}%
\settoheight{\Height}{$-4$}\settodepth{\Depth}{$-4$}\setlength{\Height}{-0.5\Height}\setlength{\Depth}{0.5\Depth}\addtolength{\Height}{\Depth}%
\put(-0.1000000,-4.0000000){\hspace*{\Width}\raisebox{\Height}{$-4$}}%
%
\special{pa    20  1181}\special{pa   -20  1181}%
\special{fp}%
\settowidth{\Width}{$-3$}\setlength{\Width}{-1\Width}%
\settoheight{\Height}{$-3$}\settodepth{\Depth}{$-3$}\setlength{\Height}{-0.5\Height}\setlength{\Depth}{0.5\Depth}\addtolength{\Height}{\Depth}%
\put(-0.1000000,-3.0000000){\hspace*{\Width}\raisebox{\Height}{$-3$}}%
%
\special{pa    20   787}\special{pa   -20   787}%
\special{fp}%
\settowidth{\Width}{$-2$}\setlength{\Width}{-1\Width}%
\settoheight{\Height}{$-2$}\settodepth{\Depth}{$-2$}\setlength{\Height}{-0.5\Height}\setlength{\Depth}{0.5\Depth}\addtolength{\Height}{\Depth}%
\put(-0.1000000,-2.0000000){\hspace*{\Width}\raisebox{\Height}{$-2$}}%
%
\special{pa    20   394}\special{pa   -20   394}%
\special{fp}%
\settowidth{\Width}{$-1$}\setlength{\Width}{-1\Width}%
\settoheight{\Height}{$-1$}\settodepth{\Depth}{$-1$}\setlength{\Height}{-0.5\Height}\setlength{\Depth}{0.5\Depth}\addtolength{\Height}{\Depth}%
\put(-0.1000000,-1.0000000){\hspace*{\Width}\raisebox{\Height}{$-1$}}%
%
\special{pa    20    -0}\special{pa   -20    -0}%
\special{fp}%
\settowidth{\Width}{$0$}\setlength{\Width}{-1\Width}%
\settoheight{\Height}{$0$}\settodepth{\Depth}{$0$}\setlength{\Height}{-0.5\Height}\setlength{\Depth}{0.5\Depth}\addtolength{\Height}{\Depth}%
\put(-0.1000000,0.0000000){\hspace*{\Width}\raisebox{\Height}{$0$}}%
%
\special{pa    20  -394}\special{pa   -20  -394}%
\special{fp}%
\settowidth{\Width}{$1$}\setlength{\Width}{-1\Width}%
\settoheight{\Height}{$1$}\settodepth{\Depth}{$1$}\setlength{\Height}{-0.5\Height}\setlength{\Depth}{0.5\Depth}\addtolength{\Height}{\Depth}%
\put(-0.1000000,1.0000000){\hspace*{\Width}\raisebox{\Height}{$1$}}%
%
\special{pa    20  -787}\special{pa   -20  -787}%
\special{fp}%
\settowidth{\Width}{$2$}\setlength{\Width}{-1\Width}%
\settoheight{\Height}{$2$}\settodepth{\Depth}{$2$}\setlength{\Height}{-0.5\Height}\setlength{\Depth}{0.5\Depth}\addtolength{\Height}{\Depth}%
\put(-0.1000000,2.0000000){\hspace*{\Width}\raisebox{\Height}{$2$}}%
%
\special{pa    20 -1181}\special{pa   -20 -1181}%
\special{fp}%
\settowidth{\Width}{$3$}\setlength{\Width}{-1\Width}%
\settoheight{\Height}{$3$}\settodepth{\Depth}{$3$}\setlength{\Height}{-0.5\Height}\setlength{\Depth}{0.5\Depth}\addtolength{\Height}{\Depth}%
\put(-0.1000000,3.0000000){\hspace*{\Width}\raisebox{\Height}{$3$}}%
%
\special{pa    20 -1575}\special{pa   -20 -1575}%
\special{fp}%
\settowidth{\Width}{$4$}\setlength{\Width}{-1\Width}%
\settoheight{\Height}{$4$}\settodepth{\Depth}{$4$}\setlength{\Height}{-0.5\Height}\setlength{\Depth}{0.5\Depth}\addtolength{\Height}{\Depth}%
\put(-0.1000000,4.0000000){\hspace*{\Width}\raisebox{\Height}{$4$}}%
%
\special{pa    20 -1969}\special{pa   -20 -1969}%
\special{fp}%
\settowidth{\Width}{$5$}\setlength{\Width}{-1\Width}%
\settoheight{\Height}{$5$}\settodepth{\Depth}{$5$}\setlength{\Height}{-0.5\Height}\setlength{\Depth}{0.5\Depth}\addtolength{\Height}{\Depth}%
\put(-0.1000000,5.0000000){\hspace*{\Width}\raisebox{\Height}{$5$}}%
%
\special{pa -2362    -0}\special{pa  2362    -0}%
\special{fp}%
\special{pa     0  2362}\special{pa     0 -2362}%
\special{fp}%
\settowidth{\Width}{$x$}\setlength{\Width}{0\Width}%
\settoheight{\Height}{$x$}\settodepth{\Depth}{$x$}\setlength{\Height}{-0.5\Height}\setlength{\Depth}{0.5\Depth}\addtolength{\Height}{\Depth}%
\put(6.0500000,0.0000000){\hspace*{\Width}\raisebox{\Height}{$x$}}%
%
\settowidth{\Width}{$y$}\setlength{\Width}{-0.5\Width}%
\settoheight{\Height}{$y$}\settodepth{\Depth}{$y$}\setlength{\Height}{\Depth}%
\put(0.0000000,6.0500000){\hspace*{\Width}\raisebox{\Height}{$y$}}%
%
\settowidth{\Width}{O}\setlength{\Width}{-1\Width}%
\settoheight{\Height}{O}\settodepth{\Depth}{O}\setlength{\Height}{-\Height}%
\put(-0.0500000,-0.0500000){\hspace*{\Width}\raisebox{\Height}{O}}%
%
\end{picture}}%}}
\end{layer}

\begin{itemize}
\item
{\normalsize\url{https://s-takato.github.io/dntnet/taisuugraph/taisuu.html}}
\item
$y=a^x$と$y=\log_a x$は\\$y=x$に関して対称
\item
このような関数を\\{\color{red}逆関数}という
\item
対数関数と指数関数は\\
\hspace*{2zw}逆関数どうし
\end{itemize}
\label{pageend}\mbox{}

\end{document}
