%%% タイトル material
\documentclass[landscape,10pt]{ujarticle}
\special{papersize=\the\paperwidth,\the\paperheight}
\usepackage{ketpic,ketlayer}
\usepackage{ketslide}
\usepackage{amsmath,amssymb}
\usepackage{bm,enumerate}
\usepackage[dvipdfmx]{graphicx}
\usepackage{color}
\definecolor{slidecolora}{cmyk}{0.98,0.13,0,0.43}
\definecolor{slidecolorb}{cmyk}{0.2,0,0,0}
\definecolor{slidecolorc}{cmyk}{0.2,0,0,0}
\definecolor{slidecolord}{cmyk}{0.2,0,0,0}
\definecolor{slidecolore}{cmyk}{0,0,0,0.5}
\definecolor{slidecolorf}{cmyk}{0,0,0,0.5}
\definecolor{slidecolori}{cmyk}{0.98,0.13,0,0.43}
\def\setthin#1{\def\thin{#1}}
\setthin{0}
\newcommand{\slidepage}[1][s]{%
\setcounter{ketpicctra}{18}%
\if#1m \setcounter{ketpicctra}{1}\fi
\hypersetup{linkcolor=black}%

\begin{layer}{118}{0}
\putnotee{122}{-\theketpicctra.05}{\small\thepage/\pageref{pageend}}
\end{layer}\hypersetup{linkcolor=blue}

}
\usepackage{emath}
\usepackage[dvipdfmx,colorlinks=true,linkcolor=blue,filecolor=blue]{hyperref}
\newcommand{\hako}[4][6]{\fbox{\raisebox{#2 mm}{$\mathstrut$}\raisebox{-#3 mm}{$\mathstrut$}\Ctab{#1 mm}{#4}}}
\def\rad{\;\mathrm{rad}}
\newcommand{\sfrac}[3][0.65]{\scalebox{#1}{$\frac{#2}{#3}$}}

\setmargin{25}{145}{15}{100}

\ketslideinit

\pagestyle{empty}

\begin{document}

\begin{layer}{120}{0}
\putnotese{0}{0}{\input{fig/slide0.tex}}
\end{layer}

\def\mainslidetitley{22}
\def\ketcletter{slidecolora}
\def\ketcbox{slidecolorb}
\def\ketdbox{slidecolorc}
\def\ketcframe{slidecolord}
\def\ketcshadow{slidecolore}
\def\ketdshadow{slidecolorf}
\def\slidetitlex{6}
\def\slidetitlesize{1.3}
\def\mketcletter{slidecolori}
\def\mketcbox{yellow}
\def\mketdbox{yellow}
\def\mketcframe{yellow}
\def\mslidetitlex{62}
\def\mslidetitlesize{2}

\color{black}
\Large\bf\boldmath
\addtocounter{page}{-1}

\renewcommand{\hako}[4][-1]{%
\setcounter{ketpicctra}{#2}%
\divide\value{ketpicctra} by 2%
\setcounter{ketpicctrb}{#3}%
\divide\value{ketpicctrb} by 2%
\setcounter{ketpicctrc}{\theketpicctrb}%
\addtocounter{ketpicctrc}{#1}%
\def\kettmp{
\begin{picture}%
(#2, #3)(0,0)%
\settowidth{\Width}{#4}\setlength{\Width}{-0.5\Width}%
\settoheight{\Height}{#4}\settodepth{\Depth}{#4}\setlength{\Height}{-0.5\Height}\setlength{\Depth}{0.5\Depth}\addtolength{\Height}{\Depth}%
\put(\theketpicctra,\theketpicctrb){\hspace*{\Width}\raisebox{\Height}{#4}}%
\end{picture}%
}%
{\unitlength=1mm%
\raisebox{-\theketpicctrc mm}{\fbox{\kettmp}}%
}
}
%%%%%%%%%%%%

%%%%%%%%%%%%%%%%%%%%

\newslide{対数関数$y=\log_a x$のグラフ}

\vspace*{18mm}

\slidepage
\begin{itemize}
\item
対数の意味は \fbox{$y=\log_a x\Longleftrightarrow x=a^y$}
\item
$y=a^x$の$x,y$を交換
\item
$x$軸と$y$軸も入れ替わる
\item
$x$軸が横,$y$軸が縦になるようにする
\item
次のURLをクリックして実行してみよう\\
\normalsize\url{https://s-takato.github.io/dntnet/taisuugraph/sisuu2taisuu.html}
\end{itemize}
%%%%%%%%%%%%

%%%%%%%%%%%%%%%%%%%%


\newslide{指数関数と対数関数}

\vspace*{18mm}

\slidepage

\begin{layer}{120}{0}
\putnotese{63}{17}{\scalebox{0.5}{\input{fig/taisuu3.tex}}}
\end{layer}

\begin{itemize}
\item
{\normalsize\url{https://s-takato.github.io/dntnet/taisuugraph/taisuu.html}}
\item
$y=a^x$と$y=\log_a x$は\\$y=x$に関して対称
\item
このような関数を\\{\color{red}逆関数}という
\item
対数関数と指数関数は\\
\hspace*{2zw}逆関数どうし
\end{itemize}

\sameslide

\vspace*{18mm}

\slidepage

\begin{layer}{120}{0}
\putnotese{63}{17}{\scalebox{0.5}{\input{fig/taisuu5.tex}}}
\end{layer}

\begin{itemize}
\item
{\normalsize\url{https://s-takato.github.io/dntnet/taisuugraph/taisuu.html}}
\item
$y=a^x$と$y=\log_a x$は\\$y=x$に関して対称
\item
このような関数を\\{\color{red}逆関数}という
\item
対数関数と指数関数は\\
\hspace*{2zw}逆関数どうし
\end{itemize}

\sameslide

\vspace*{18mm}

\slidepage

\begin{layer}{120}{0}
\putnotese{63}{17}{\scalebox{0.5}{\input{fig/taisuu10.tex}}}
\end{layer}

\begin{itemize}
\item
{\normalsize\url{https://s-takato.github.io/dntnet/taisuugraph/taisuu.html}}
\item
$y=a^x$と$y=\log_a x$は\\$y=x$に関して対称
\item
このような関数を\\{\color{red}逆関数}という
\item
対数関数と指数関数は\\
\hspace*{2zw}逆関数どうし
\end{itemize}

\sameslide

\vspace*{18mm}

\slidepage

\begin{layer}{120}{0}
\putnotese{63}{17}{\scalebox{0.5}{\input{fig/taisuum2.tex}}}
\end{layer}

\begin{itemize}
\item
{\normalsize\url{https://s-takato.github.io/dntnet/taisuugraph/taisuu.html}}
\item
$y=a^x$と$y=\log_a x$は\\$y=x$に関して対称
\item
このような関数を\\{\color{red}逆関数}という
\item
対数関数と指数関数は\\
\hspace*{2zw}逆関数どうし
\end{itemize}
\label{pageend}\mbox{}

\end{document}
